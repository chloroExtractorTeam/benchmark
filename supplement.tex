\documentclass[oneside,10pt,A4]{article}

%%% Load packages
%\usepackage{amsthm,amsmath}
%\RequirePackage{natbib}
%\RequirePackage[authoryear]{natbib}% uncomment this for author-year bibliography
\RequirePackage{hyperref}
\usepackage[utf8]{inputenc} %unicode support
%\usepackage[applemac]{inputenc} %applemac support if unicode package fails
%\usepackage[latin1]{inputenc} %UNIX support if unicode package fails
\usepackage{graphicx}

\usepackage[noabbrev]{cleveref}
\usepackage{siunitx}
\DeclareSIUnit \basepair{bp}
\DeclareSIUnit \byte{B}
\DeclareSIPrefix\mebi{Mi}{10}
\DeclareSIPrefix\gibi{Gi}{10}

\usepackage{biocon}
\newplant{At}{genus=Arabidopsis,epithet=thaliana}

%%%%%%%%%%%%%%%%%%%%%%%%%%%%%%%%%%%%%%%%%%%%%%%%%
%%                                             %%
%%  If you wish to display your graphics for   %%
%%  your own use using includegraphic or       %%
%%  includegraphics, then comment out the      %%
%%  following two lines of code.               %%
%%  NB: These line *must* be included when     %%
%%  submitting to BMC.                         %%
%%  All figure files must be submitted as      %%
%%  separate graphics through the BMC          %%
%%  submission process, not included in the    %%
%%  submitted article.                         %%
%%                                             %%
%%%%%%%%%%%%%%%%%%%%%%%%%%%%%%%%%%%%%%%%%%%%%%%%%


%\def\includegraphic{}
%\def\includegraphics{}

\usepackage{todonotes}
\setlength{\marginparwidth}{3cm}

\newcounter{todocounter}
\newcommand{\ff}[1]
{\stepcounter{todocounter}
 \todo[color=blue!40,author=For Frank]{\thetodocounter: #1}
 }
\newcommand{\ak}[1]
{\stepcounter{todocounter}
 \todo[color=green!40,author=Arthur]{\thetodocounter: #1}
 }


%%% Put your definitions there:
\newcommand{\formatprogramnames}[1]{\texttt{#1}}
\newcommand{\ce}{\formatprogramnames{chloroExtractor}}
\newcommand{\oa}{\formatprogramnames{ORG.Asm}}
\newcommand{\fp}{\formatprogramnames{Fast-Plast}}
\newcommand{\ioga}{\formatprogramnames{IOGA}}
\newcommand{\np}{\formatprogramnames{NOVOPlasty}}
\newcommand{\go}{\formatprogramnames{GetOrganelle}}
\newcommand{\cassp}{\formatprogramnames{Chloroplast assembly protocol}}

\newcommand{\zenododataset}{\cite{zenododataset}}
\newcommand{\zenodorepo}{\cite{zenodorepo}}

\newcommand{\genename}[1]{\textit{#1}}

% Niklas table
%\newcommand{\ok}{\textcolor[rgb]{0.7,0.7,0}{\textsc{\bfseries{}OKAY}}}
%\newcommand{\bad}{\textcolor{red}{\textsc{\bfseries{}BAD}}}
%\newcommand{\good}{\textcolor[rgb]{0,0.6,0}{\textsc{\bfseries{}GOOD}}}
\newcommand{\ok}{\textsc{Okay}}
\newcommand{\bad}{\textsc{Bad}}
\newcommand{\good}{\textsc{Good}}


% List of docker images
\newcommand{\docker}[2]{\href{https://cloud.docker.com/u/chloroextractorteam/repository/docker/chloroextractorteam/#1}{chloroextracteam/#1:#2}}
\newcommand{\dockerce}{\docker{benchmark\_chloroextractor}{v2.0.0}}
\newcommand{\dockercesha}{\texttt{8f49e03424b37e699c5fea9391f79dbb9f2d0dc550ce86c4c700f39deec2dacd}}
\newcommand{\dockeroa}{\docker{benchmark\_org-asm}{v2.0.0}}
\newcommand{\dockeroasha}{\texttt{ff83677c97b7c4e346191b9e04a5162eeea2a8dfd9e60ff84067d33747868f60}}
\newcommand{\dockerfp}{\docker{benchmark\_fastplast}{v2.0.0}}
\newcommand{\dockerfpsha}{\texttt{a3d06a610f8340ba49c3ff3b27342534e2a5348b17caa4fd3c3f3d327243a272}}
\newcommand{\dockerioga}{\docker{benchmark\_ioga}{v2.0.0}}
\newcommand{\dockeriogasha}{\texttt{4698a21c343c60290bbf16811165a654ff01e8fddeb41cd75d0771b7f45968c0}}
\newcommand{\dockernp}{\docker{benchmark\_novoplasty}{v2.0.0}}
\newcommand{\dockernpsha}{\texttt{106387bad4e8e5c53fb9d4c3bdd60cdddf05a47b52a38697369eeabb947c547c}}
\newcommand{\dockergo}{\docker{benchmark\_getorganelle}{v2.0.0}}
\newcommand{\dockergosha}{\texttt{2ed3a464a82025a196ea56c649d0b0a3472cef76994bb065d56054d93298d956}}
\newcommand{\dockercassp}{\docker{benchmark\_chloroplast\_assembly\_protocol}{v2.0.0}}
\newcommand{\dockercasspsha}{\texttt{d10d2317103d91fadec4cb1d5466ac109f23176ad74b6f44f0179175f050155d}}

\usepackage{longtable}
\usepackage{pdflscape}
\usepackage{capt-of}

\renewcommand{\thetable}{S\arabic{table}}%
\renewcommand{\thefigure}{S\arabic{figure}}
%%% Begin ...
\begin{document}

\listoftables

\begin{landscape}
\setlength{\LTcapwidth}{1.5\textwidth}
\begin{longtable}{cccrrcc}
\caption[Summary real data set]{\textbf{Summary real data set.} List of real data sets used during our benchmark study. The tables was sorted by species name. Additionally, the table contains the accession for the Sequencing Read Archive (SRA Acc) and the accession of the used reference (Ref Acc). Moreover, the original number of read pairs (read pairs), the overall number of sequenced nucleotides (base pairs), and the mean read length (read len.) is given. Finally, the last column contains the information, if that data set was used for the check of the consistency.} \\
\hline
Taxon & SRA Acc & Ref Acc & read pairs & base pairs (bp) & read len. (bp) & Consistency check \\
\hline
\endfirsthead
\multicolumn{7}{c}{\tablename\ \thetable\ -- \textit{Continued from previous page}}\\[0.5em]
Taxon & SRA Acc & Ref Acc & read pairs & base pairs (bp) & read len. (bp) & Consistency check \\ \hline
\endhead
\hline
\multicolumn{7}{r}{Continued on next page}\\
\endfoot
\endlastfoot
\textit{Abelmoschus esculentus} & \href{https://trace.ncbi.nlm.nih.gov/Traces/sra/?run=SRR5812498}{SRR5812498} & \href{https://www.ncbi.nlm.nih.gov/nuccore/NC_035234}{NC\_035234} & \num{9028889} & \num{2726724478} & \num{151} & Yes \\
\textit{Abies sibirica} & \href{https://trace.ncbi.nlm.nih.gov/Traces/sra/?run=ERR268415}{ERR268415} & \href{https://www.ncbi.nlm.nih.gov/nuccore/NC_035067}{NC\_035067} & \num{157369451} & \num{31473890200} & \num{100} & NO \\
\textit{Actinidia chinensis} & \href{https://trace.ncbi.nlm.nih.gov/Traces/sra/?run=DRR083750}{DRR083750} & \href{https://www.ncbi.nlm.nih.gov/nuccore/NC_026690}{NC\_026690} & \num{1974014} & \num{574124022} & \num{145} & Yes \\
\textit{Aegilops sharonensis} & \href{https://trace.ncbi.nlm.nih.gov/Traces/sra/?run=ERR359708}{ERR359708} & \href{https://www.ncbi.nlm.nih.gov/nuccore/NC_024816}{NC\_024816} & \num{179478974} & \num{36254752748} & \num{101} & NO \\
\textit{Aegilops speltoides} & \href{https://trace.ncbi.nlm.nih.gov/Traces/sra/?run=ERR424849}{ERR424849} & \href{https://www.ncbi.nlm.nih.gov/nuccore/NC_022135}{NC\_022135} & \num{173225639} & \num{34991579078} & \num{101} & NO \\
\textit{Aegilops tauschii} & \href{https://trace.ncbi.nlm.nih.gov/Traces/sra/?run=SRR091633}{SRR091633} & \href{https://www.ncbi.nlm.nih.gov/nuccore/NC_022133}{NC\_022133} & \num{114178484} & \num{23064053768} & \num{101} & NO \\
\textit{Ajuga reptans} & \href{https://trace.ncbi.nlm.nih.gov/Traces/sra/?run=SRR6940062}{SRR6940062} & \href{https://www.ncbi.nlm.nih.gov/nuccore/NC_023102}{NC\_023102} & \num{16864855} & \num{5059456500} & \num{150} & NO \\
\textit{Aldrovanda vesiculosa} & \href{https://trace.ncbi.nlm.nih.gov/Traces/sra/?run=SRR7072768}{SRR7072768} & \href{https://www.ncbi.nlm.nih.gov/nuccore/NC_035416}{NC\_035416} & \num{3510392} & \num{1060138384} & \num{151} & Yes \\
\textit{Allium cepa} & \href{https://trace.ncbi.nlm.nih.gov/Traces/sra/?run=SRR1686960}{SRR1686960} & \href{https://www.ncbi.nlm.nih.gov/nuccore/NC_024813}{NC\_024813} & \num{90042453} & \num{27099552373} & \num{150} & Yes \\
\textit{Allium sativum} & \href{https://trace.ncbi.nlm.nih.gov/Traces/sra/?run=SRR5602598}{SRR5602598} & \href{https://www.ncbi.nlm.nih.gov/nuccore/NC_031829}{NC\_031829} & \num{1270060} & \num{748290736} & \num{295} & NO \\
\textit{Alloteropsis angusta} & \href{https://trace.ncbi.nlm.nih.gov/Traces/sra/?run=SRR7528995}{SRR7528995} & \href{https://www.ncbi.nlm.nih.gov/nuccore/NC_027951}{NC\_027951} & \num{18695483} & \num{9347741500} & \num{250} & NO \\
\textit{Alloteropsis cimicina} & \href{https://trace.ncbi.nlm.nih.gov/Traces/sra/?run=SRR7529015}{SRR7529015} & \href{https://www.ncbi.nlm.nih.gov/nuccore/NC_027952}{NC\_027952} & \num{118496918} & \num{59248459000} & \num{250} & Yes \\
\textit{Alloteropsis paniculata} & \href{https://trace.ncbi.nlm.nih.gov/Traces/sra/?run=SRR4051980}{SRR4051980} & \href{https://www.ncbi.nlm.nih.gov/nuccore/NC_032078}{NC\_032078} & \num{4048486} & \num{796542955} & \num{98} & NO \\
\textit{Alloteropsis semialata} & \href{https://trace.ncbi.nlm.nih.gov/Traces/sra/?run=SRR7528994}{SRR7528994} & \href{https://www.ncbi.nlm.nih.gov/nuccore/NC_027824}{NC\_027824} & \num{26426678} & \num{13213339000} & \num{250} & Yes \\
\textit{Aloysia citrodora} & \href{https://trace.ncbi.nlm.nih.gov/Traces/sra/?run=SRR5602597}{SRR5602597} & \href{https://www.ncbi.nlm.nih.gov/nuccore/NC_034695}{NC\_034695} & \num{1423723} & \num{844386488} & \num{297} & NO \\
\textit{Althaea officinalis} & \href{https://trace.ncbi.nlm.nih.gov/Traces/sra/?run=SRR5602596}{SRR5602596} & \href{https://www.ncbi.nlm.nih.gov/nuccore/NC_034701}{NC\_034701} & \num{1409711} & \num{833584333} & \num{296} & NO \\
\textit{Amaranthus hypochondriacus} & \href{https://trace.ncbi.nlm.nih.gov/Traces/sra/?run=SRR3399331}{SRR3399331} & \href{https://www.ncbi.nlm.nih.gov/nuccore/NC_030770}{NC\_030770} & \num{6509} & \num{3113188} & \num{239} & NO \\
\textit{Ammopiptanthus nanus} & \href{https://trace.ncbi.nlm.nih.gov/Traces/sra/?run=SRR6175451}{SRR6175451} & \href{https://www.ncbi.nlm.nih.gov/nuccore/NC_034743}{NC\_034743} & \num{189485569} & \num{56845670700} & \num{150} & NO \\
\textit{Anacardium occidentale} & \href{https://trace.ncbi.nlm.nih.gov/Traces/sra/?run=SRR5812497}{SRR5812497} & \href{https://www.ncbi.nlm.nih.gov/nuccore/NC_035235}{NC\_035235} & \num{27997137} & \num{7055278524} & \num{126} & NO \\
\textit{Apostasia odorata} & \href{https://trace.ncbi.nlm.nih.gov/Traces/sra/?run=SRR6037860}{SRR6037860} & \href{https://www.ncbi.nlm.nih.gov/nuccore/NC_030722}{NC\_030722} & \num{48546649} & \num{14563994700} & \num{150} & Yes \\
\textit{Arabidopsis arenicola} & \href{https://trace.ncbi.nlm.nih.gov/Traces/sra/?run=DRR054584}{DRR054584} & \href{https://www.ncbi.nlm.nih.gov/nuccore/NC_030346}{NC\_030346} & \num{31543276} & \num{6371741752} & \num{101} & Yes \\
\textit{Arabidopsis arenosa} & \href{https://trace.ncbi.nlm.nih.gov/Traces/sra/?run=SRR4128972}{SRR4128972} & \href{https://www.ncbi.nlm.nih.gov/nuccore/NC_029334}{NC\_029334} & \num{129638900} & \num{38891670000} & \num{150} & Yes \\
\textit{Arabidopsis cebennensis} & \href{https://trace.ncbi.nlm.nih.gov/Traces/sra/?run=SRR2040775}{SRR2040775} & \href{https://www.ncbi.nlm.nih.gov/nuccore/NC_029335}{NC\_029335} & \num{20358302} & \num{4112377004} & \num{101} & Yes \\
\textit{Arabidopsis croatica} & \href{https://trace.ncbi.nlm.nih.gov/Traces/sra/?run=SRR7637319}{SRR7637319} & \href{https://www.ncbi.nlm.nih.gov/nuccore/NC_030347}{NC\_030347} & \num{14265071} & \num{3573463604} & \num{125} & NO \\
\textit{Arabidopsis halleri} & \href{https://trace.ncbi.nlm.nih.gov/Traces/sra/?run=SRR8040824}{SRR8040824} & \href{https://www.ncbi.nlm.nih.gov/nuccore/NC_034366}{NC\_034366} & \num{128303531} & \num{38747666362} & \num{151} & Yes \\
\textit{Arabidopsis lyrata subsp. lyrata} & \href{https://trace.ncbi.nlm.nih.gov/Traces/sra/?run=SRR1705621}{SRR1705621} & \href{https://www.ncbi.nlm.nih.gov/nuccore/NC_034379}{NC\_034379} & \num{16771948} & \num{4796683328} & \num{143} & NO \\
\textit{Arabidopsis lyrata} & \href{https://trace.ncbi.nlm.nih.gov/Traces/sra/?run=SRR8157563}{SRR8157563} & \href{https://www.ncbi.nlm.nih.gov/nuccore/NC_034365}{NC\_034365} & \num{46678588} & \num{9429074776} & \num{101} & NO \\
\textit{Arabidopsis neglecta} & \href{https://trace.ncbi.nlm.nih.gov/Traces/sra/?run=SRR2040831}{SRR2040831} & \href{https://www.ncbi.nlm.nih.gov/nuccore/NC_030348}{NC\_030348} & \num{52410983} & \num{10482196600} & \num{100} & Yes \\
\textit{Arabidopsis pedemontana} & \href{https://trace.ncbi.nlm.nih.gov/Traces/sra/?run=SRR2040802}{SRR2040802} & \href{https://www.ncbi.nlm.nih.gov/nuccore/NC_029336}{NC\_029336} & \num{28098774} & \num{5675952348} & \num{101} & NO \\
\textit{Arabidopsis petrogena} & \href{https://trace.ncbi.nlm.nih.gov/Traces/sra/?run=SRR2040807}{SRR2040807} & \href{https://www.ncbi.nlm.nih.gov/nuccore/NC_030349}{NC\_030349} & \num{32189424} & \num{6502263648} & \num{101} & Yes \\
\textit{Arabidopsis suecica} & \href{https://trace.ncbi.nlm.nih.gov/Traces/sra/?run=SRR2084154}{SRR2084154} & \href{https://www.ncbi.nlm.nih.gov/nuccore/NC_030350}{NC\_030350} & \num{158377327} & \num{31675465400} & \num{100} & NO \\
\textit{Arabidopsis thaliana} & \href{https://trace.ncbi.nlm.nih.gov/Traces/sra/?run=SRR3340908}{SRR3340908} & \href{https://www.ncbi.nlm.nih.gov/nuccore/NC_000932}{NC\_000932} & \num{6829274} & \num{4097564400} & \num{300} & NO \\
\textit{Arabidopsis umezawana} & \href{https://trace.ncbi.nlm.nih.gov/Traces/sra/?run=SRR2040810}{SRR2040810} & \href{https://www.ncbi.nlm.nih.gov/nuccore/NC_030351}{NC\_030351} & \num{24335095} & \num{4915689190} & \num{101} & NO \\
\textit{Arabis alpina} & \href{https://trace.ncbi.nlm.nih.gov/Traces/sra/?run=SRR7880736}{SRR7880736} & \href{https://www.ncbi.nlm.nih.gov/nuccore/NC_023367}{NC\_023367} & \num{12603925} & \num{3126123117} & \num{124} & Yes \\
\textit{Artemisia annua} & \href{https://trace.ncbi.nlm.nih.gov/Traces/sra/?run=SRR5602595}{SRR5602595} & \href{https://www.ncbi.nlm.nih.gov/nuccore/NC_034683}{NC\_034683} & \num{665200} & \num{330541594} & \num{248} & Yes \\
\textit{Arundo plinii} & \href{https://trace.ncbi.nlm.nih.gov/Traces/sra/?run=SRR4319202}{SRR4319202} & \href{https://www.ncbi.nlm.nih.gov/nuccore/NC_034652}{NC\_034652} & \num{10180635} & \num{6099219728} & \num{300} & Yes \\
\textit{Asclepias nivea} & \href{https://trace.ncbi.nlm.nih.gov/Traces/sra/?run=SRR934044}{SRR934044} & \href{https://www.ncbi.nlm.nih.gov/nuccore/NC_022431}{NC\_022431} & \num{1637178} & \num{311063820} & \num{95} & NO \\
\textit{Asclepias syriaca} & \href{https://trace.ncbi.nlm.nih.gov/Traces/sra/?run=SRR934060}{SRR934060} & \href{https://www.ncbi.nlm.nih.gov/nuccore/NC_022432}{NC\_022432} & \num{46704483} & \num{9434305566} & \num{101} & NO \\
\textit{Asparagus officinalis} & \href{https://trace.ncbi.nlm.nih.gov/Traces/sra/?run=DRR056675}{DRR056675} & \href{https://www.ncbi.nlm.nih.gov/nuccore/NC_034777}{NC\_034777} & \num{83276068} & \num{16821765736} & \num{101} & Yes \\
\textit{Astragalus mongholicus} & \href{https://trace.ncbi.nlm.nih.gov/Traces/sra/?run=SRR3938254}{SRR3938254} & \href{https://www.ncbi.nlm.nih.gov/nuccore/NC_029828}{NC\_029828} & \num{7500181} & \num{1514903465} & \num{101} & Yes \\
\textit{Avena sativa} & \href{https://trace.ncbi.nlm.nih.gov/Traces/sra/?run=SRR6056489}{SRR6056489} & \href{https://www.ncbi.nlm.nih.gov/nuccore/NC_027468}{NC\_027468} & \num{115104823} & \num{57552411500} & \num{250} & NO \\
\textit{Averrhoa carambola} & \href{https://trace.ncbi.nlm.nih.gov/Traces/sra/?run=ERR2799539}{ERR2799539} & \href{https://www.ncbi.nlm.nih.gov/nuccore/NC_033350}{NC\_033350} & \num{1023284} & \num{309031768} & \num{151} & NO \\
\textit{Betula nana} & \href{https://trace.ncbi.nlm.nih.gov/Traces/sra/?run=ERR2026268}{ERR2026268} & \href{https://www.ncbi.nlm.nih.gov/nuccore/NC_033978}{NC\_033978} & \num{45426355} & \num{13797661576} & \num{152} & NO \\
\textit{Boswellia sacra} & \href{https://trace.ncbi.nlm.nih.gov/Traces/sra/?run=SRR5602594}{SRR5602594} & \href{https://www.ncbi.nlm.nih.gov/nuccore/NC_029420}{NC\_029420} & \num{2012947} & \num{1199195198} & \num{298} & Yes \\
\textit{Botryococcus braunii} & \href{https://trace.ncbi.nlm.nih.gov/Traces/sra/?run=SRR3721649}{SRR3721649} & \href{https://www.ncbi.nlm.nih.gov/nuccore/NC_025545}{NC\_025545} & \num{137313108} & \num{41193932400} & \num{150} & NO \\
\textit{Brachypodium distachyon} & \href{https://trace.ncbi.nlm.nih.gov/Traces/sra/?run=SRR4159530}{SRR4159530} & \href{https://www.ncbi.nlm.nih.gov/nuccore/NC_011032}{NC\_011032} & \num{51410001} & \num{15525820302} & \num{151} & Yes \\
\textit{Brassica juncea} & \href{https://trace.ncbi.nlm.nih.gov/Traces/sra/?run=SRR2057910}{SRR2057910} & \href{https://www.ncbi.nlm.nih.gov/nuccore/NC_028272}{NC\_028272} & \num{54890460} & \num{11087872920} & \num{101} & NO \\
\textit{Brassica napus} & \href{https://trace.ncbi.nlm.nih.gov/Traces/sra/?run=SRR6856349}{SRR6856349} & \href{https://www.ncbi.nlm.nih.gov/nuccore/NC_016734}{NC\_016734} & \num{34913922} & \num{10544004444} & \num{151} & NO \\
\textit{Brassica nigra} & \href{https://trace.ncbi.nlm.nih.gov/Traces/sra/?run=SRR2054762}{SRR2054762} & \href{https://www.ncbi.nlm.nih.gov/nuccore/NC_030450}{NC\_030450} & \num{57198732} & \num{11554143864} & \num{101} & Yes \\
\textit{Bupleurum latissimum} & \href{https://trace.ncbi.nlm.nih.gov/Traces/sra/?run=SRR6048019}{SRR6048019} & \href{https://www.ncbi.nlm.nih.gov/nuccore/NC_033346}{NC\_033346} & \num{5702505} & \num{3432908010} & \num{301} & Yes \\
\textit{Cajanus cajan} & \href{https://trace.ncbi.nlm.nih.gov/Traces/sra/?run=SRR073338}{SRR073338} & \href{https://www.ncbi.nlm.nih.gov/nuccore/NC_031429}{NC\_031429} & \num{132406303} & \num{26746073206} & \num{101} & Yes \\
\textit{Camelina sativa} & \href{https://trace.ncbi.nlm.nih.gov/Traces/sra/?run=SRR2082664}{SRR2082664} & \href{https://www.ncbi.nlm.nih.gov/nuccore/NC_029337}{NC\_029337} & \num{10643512} & \num{2149989424} & \num{101} & Yes \\
\textit{Camellia sinensis} & \href{https://trace.ncbi.nlm.nih.gov/Traces/sra/?run=SRR5894861}{SRR5894861} & \href{https://www.ncbi.nlm.nih.gov/nuccore/NC_020019}{NC\_020019} & \num{4975058} & \num{2985034800} & \num{300} & NO \\
\textit{Cannabis sativa} & \href{https://trace.ncbi.nlm.nih.gov/Traces/sra/?run=SRR7285294}{SRR7285294} & \href{https://www.ncbi.nlm.nih.gov/nuccore/NC_026562}{NC\_026562} & \num{162968810} & \num{42761008291} & \num{131} & Yes \\
\textit{Capsella bursa-pastoris} & \href{https://trace.ncbi.nlm.nih.gov/Traces/sra/?run=SRR5412136}{SRR5412136} & \href{https://www.ncbi.nlm.nih.gov/nuccore/NC_009270}{NC\_009270} & \num{15148015} & \num{9391769300} & \num{310} & NO \\
\textit{Capsella grandiflora} & \href{https://trace.ncbi.nlm.nih.gov/Traces/sra/?run=SRR1508428}{SRR1508428} & \href{https://www.ncbi.nlm.nih.gov/nuccore/NC_028517}{NC\_028517} & \num{98396864} & \num{19679372800} & \num{100} & NO \\
\textit{Capsella rubella} & \href{https://trace.ncbi.nlm.nih.gov/Traces/sra/?run=ERR636124}{ERR636124} & \href{https://www.ncbi.nlm.nih.gov/nuccore/NC_027693}{NC\_027693} & \num{40236079} & \num{12151295858} & \num{151} & NO \\
\textit{Capsicum annuum} & \href{https://trace.ncbi.nlm.nih.gov/Traces/sra/?run=SRR1982881}{SRR1982881} & \href{https://www.ncbi.nlm.nih.gov/nuccore/NC_018552}{NC\_018552} & \num{10804729} & \num{1966460678} & \num{91} & Yes \\
\textit{Carica papaya} & \href{https://trace.ncbi.nlm.nih.gov/Traces/sra/?run=DRR019500}{DRR019500} & \href{https://www.ncbi.nlm.nih.gov/nuccore/NC_010323}{NC\_010323} & \num{103259953} & \num{22510669754} & \num{109} & Yes \\
\textit{Carnegiea gigantea} & \href{https://trace.ncbi.nlm.nih.gov/Traces/sra/?run=SRR5036293}{SRR5036293} & \href{https://www.ncbi.nlm.nih.gov/nuccore/NC_027618}{NC\_027618} & \num{12642777} & \num{7585666200} & \num{300} & NO \\
\textit{Carthamus tinctorius} & \href{https://trace.ncbi.nlm.nih.gov/Traces/sra/?run=SRR2154065}{SRR2154065} & \href{https://www.ncbi.nlm.nih.gov/nuccore/NC_030783}{NC\_030783} & \num{9648870} & \num{1949071740} & \num{101} & NO \\
\textit{Castanea mollissima} & \href{https://trace.ncbi.nlm.nih.gov/Traces/sra/?run=SRR8731963}{SRR8731963} & \href{https://www.ncbi.nlm.nih.gov/nuccore/NC_014674}{NC\_014674} & \num{118123542} & \num{35437062600} & \num{150} & Yes \\
\textit{Catharanthus roseus} & \href{https://trace.ncbi.nlm.nih.gov/Traces/sra/?run=SRR5217671}{SRR5217671} & \href{https://www.ncbi.nlm.nih.gov/nuccore/NC_021423}{NC\_021423} & \num{23799598} & \num{7139879400} & \num{150} & Yes \\
\textit{Cenchrus americanus} & \href{https://trace.ncbi.nlm.nih.gov/Traces/sra/?run=SRR5204424}{SRR5204424} & \href{https://www.ncbi.nlm.nih.gov/nuccore/NC_024171}{NC\_024171} & \num{149257429} & \num{36289690596} & \num{122} & Yes \\
\textit{Centaurea diffusa} & \href{https://trace.ncbi.nlm.nih.gov/Traces/sra/?run=SRR2729212}{SRR2729212} & \href{https://www.ncbi.nlm.nih.gov/nuccore/NC_024286}{NC\_024286} & \num{69630024} & \num{13926004800} & \num{100} & Yes \\
\textit{Chenopodium quinoa} & \href{https://trace.ncbi.nlm.nih.gov/Traces/sra/?run=SRR5938314}{SRR5938314} & \href{https://www.ncbi.nlm.nih.gov/nuccore/NC_034949}{NC\_034949} & \num{41331617} & \num{20748471734} & \num{251} & NO \\
\textit{Chlamydomonas reinhardtii} & \href{https://trace.ncbi.nlm.nih.gov/Traces/sra/?run=SRR4235199}{SRR4235199} & \href{https://www.ncbi.nlm.nih.gov/nuccore/NC_005353}{NC\_005353} & \num{4795246} & \num{1448164292} & \num{151} & Yes \\
\textit{Chlorella sorokiniana} & \href{https://trace.ncbi.nlm.nih.gov/Traces/sra/?run=SRR4292644}{SRR4292644} & \href{https://www.ncbi.nlm.nih.gov/nuccore/NC_023835}{NC\_023835} & \num{24148422} & \num{4877981244} & \num{101} & NO \\
\textit{Chlorella variabilis} & \href{https://trace.ncbi.nlm.nih.gov/Traces/sra/?run=SRR7012756}{SRR7012756} & \href{https://www.ncbi.nlm.nih.gov/nuccore/NC_015359}{NC\_015359} & \num{16059428} & \num{4837652469} & \num{151} & NO \\
\textit{Chromochloris zofingiensis} & \href{https://trace.ncbi.nlm.nih.gov/Traces/sra/?run=SRR5310953}{SRR5310953} & \href{https://www.ncbi.nlm.nih.gov/nuccore/NC_029672}{NC\_029672} & \num{63205006} & \num{12641001200} & \num{100} & Yes \\
\textit{Chrysobalanus icaco} & \href{https://trace.ncbi.nlm.nih.gov/Traces/sra/?run=SRR1179655}{SRR1179655} & \href{https://www.ncbi.nlm.nih.gov/nuccore/NC_024061}{NC\_024061} & \num{5544469} & \num{1119982738} & \num{101} & Yes \\
\textit{Cicer arietinum} & \href{https://trace.ncbi.nlm.nih.gov/Traces/sra/?run=SRR5183095}{SRR5183095} & \href{https://www.ncbi.nlm.nih.gov/nuccore/NC_011163}{NC\_011163} & \num{37577869} & \num{6764016420} & \num{90} & Yes \\
\textit{Cinnamomum verum} & \href{https://trace.ncbi.nlm.nih.gov/Traces/sra/?run=SRR5812493}{SRR5812493} & \href{https://www.ncbi.nlm.nih.gov/nuccore/NC_035236}{NC\_035236} & \num{4931460} & \num{1489300920} & \num{151} & NO \\
\textit{Citrullus lanatus} & \href{https://trace.ncbi.nlm.nih.gov/Traces/sra/?run=SRR8523014}{SRR8523014} & \href{https://www.ncbi.nlm.nih.gov/nuccore/NC_032008}{NC\_032008} & \num{27574511} & \num{16544706600} & \num{300} & Yes \\
\textit{Citrus aurantiifolia} & \href{https://trace.ncbi.nlm.nih.gov/Traces/sra/?run=SRR6188466}{SRR6188466} & \href{https://www.ncbi.nlm.nih.gov/nuccore/NC_024929}{NC\_024929} & \num{132426000} & \num{26485200000} & \num{100} & NO \\
\textit{Citrus limon} & \href{https://trace.ncbi.nlm.nih.gov/Traces/sra/?run=SRR5602593}{SRR5602593} & \href{https://www.ncbi.nlm.nih.gov/nuccore/NC_034690}{NC\_034690} & \num{1362792} & \num{666398607} & \num{244} & NO \\
\textit{Citrus maxima} & \href{https://trace.ncbi.nlm.nih.gov/Traces/sra/?run=SRR3407098}{SRR3407098} & \href{https://www.ncbi.nlm.nih.gov/nuccore/NC_034290}{NC\_034290} & \num{149418786} & \num{47814011520} & \num{160} & NO \\
\textit{Citrus reticulata} & \href{https://trace.ncbi.nlm.nih.gov/Traces/sra/?run=SRR3747540}{SRR3747540} & \href{https://www.ncbi.nlm.nih.gov/nuccore/NC_034671}{NC\_034671} & \num{53225257} & \num{15967577100} & \num{150} & Yes \\
\textit{Citrus sinensis} & \href{https://trace.ncbi.nlm.nih.gov/Traces/sra/?run=SRR3927459}{SRR3927459} & \href{https://www.ncbi.nlm.nih.gov/nuccore/NC_008334}{NC\_008334} & \num{46781458} & \num{14034437400} & \num{150} & NO \\
\textit{Coffea arabica} & \href{https://trace.ncbi.nlm.nih.gov/Traces/sra/?run=SRR5602572}{SRR5602572} & \href{https://www.ncbi.nlm.nih.gov/nuccore/NC_008535}{NC\_008535} & \num{2532608} & \num{1494738062} & \num{295} & NO \\
\textit{Coffea canephora} & \href{https://trace.ncbi.nlm.nih.gov/Traces/sra/?run=ERR321701}{ERR321701} & \href{https://www.ncbi.nlm.nih.gov/nuccore/NC_030053}{NC\_030053} & \num{33362762} & \num{7206356592} & \num{108} & Yes \\
\textit{Corymbia torelliana} & \href{https://trace.ncbi.nlm.nih.gov/Traces/sra/?run=SRR4063804}{SRR4063804} & \href{https://www.ncbi.nlm.nih.gov/nuccore/NC_028410}{NC\_028410} & \num{66804809} & \num{20041442700} & \num{150} & Yes \\
\textit{Couepia guianensis} & \href{https://trace.ncbi.nlm.nih.gov/Traces/sra/?run=SRR1179648}{SRR1179648} & \href{https://www.ncbi.nlm.nih.gov/nuccore/NC_024063}{NC\_024063} & \num{7411159} & \num{1497054118} & \num{101} & Yes \\
\textit{Cucumis melo subsp. melo} & \href{https://trace.ncbi.nlm.nih.gov/Traces/sra/?run=ERR246536}{ERR246536} & \href{https://www.ncbi.nlm.nih.gov/nuccore/NC_015983}{NC\_015983} & \num{33207205} & \num{10094990320} & \num{152} & NO \\
\textit{Cucumis sativus} & \href{https://trace.ncbi.nlm.nih.gov/Traces/sra/?run=SRR6490504}{SRR6490504} & \href{https://www.ncbi.nlm.nih.gov/nuccore/NC_007144}{NC\_007144} & \num{35348114} & \num{10646772194} & \num{151} & Yes \\
\textit{Cunninghamia lanceolata} & \href{https://trace.ncbi.nlm.nih.gov/Traces/sra/?run=ERR2799542}{ERR2799542} & \href{https://www.ncbi.nlm.nih.gov/nuccore/NC_021437}{NC\_021437} & \num{1062614} & \num{320909428} & \num{151} & Yes \\
\textit{Cymbidium ensifolium} & \href{https://trace.ncbi.nlm.nih.gov/Traces/sra/?run=SRR6117755}{SRR6117755} & \href{https://www.ncbi.nlm.nih.gov/nuccore/NC_028525}{NC\_028525} & \num{12766736} & \num{7685575072} & \num{301} & Yes \\
\textit{Cymbidium kanran} & \href{https://trace.ncbi.nlm.nih.gov/Traces/sra/?run=SRR6117756}{SRR6117756} & \href{https://www.ncbi.nlm.nih.gov/nuccore/NC_029711}{NC\_029711} & \num{5534224} & \num{3331602848} & \num{301} & NO \\
\textit{Cymbidium lancifolium} & \href{https://trace.ncbi.nlm.nih.gov/Traces/sra/?run=SRR6117757}{SRR6117757} & \href{https://www.ncbi.nlm.nih.gov/nuccore/NC_029712}{NC\_029712} & \num{5856776} & \num{3525779152} & \num{301} & NO \\
\textit{Cymbidium macrorhizon} & \href{https://trace.ncbi.nlm.nih.gov/Traces/sra/?run=SRR6117754}{SRR6117754} & \href{https://www.ncbi.nlm.nih.gov/nuccore/NC_029713}{NC\_029713} & \num{5777841} & \num{3478260282} & \num{301} & NO \\
\textit{Cynodon dactylon} & \href{https://trace.ncbi.nlm.nih.gov/Traces/sra/?run=SRR5457035}{SRR5457035} & \href{https://www.ncbi.nlm.nih.gov/nuccore/NC_034680}{NC\_034680} & \num{121672} & \num{65496192} & \num{269} & Yes \\
\textit{Cytinus hypocistis} & \href{https://trace.ncbi.nlm.nih.gov/Traces/sra/?run=ERR964904}{ERR964904} & \href{https://www.ncbi.nlm.nih.gov/nuccore/NC_031150}{NC\_031150} & \num{8664415} & \num{1750211830} & \num{101} & NO \\
\textit{Dactylis glomerata} & \href{https://trace.ncbi.nlm.nih.gov/Traces/sra/?run=SRR5236602}{SRR5236602} & \href{https://www.ncbi.nlm.nih.gov/nuccore/NC_027473}{NC\_027473} & \num{131404213} & \num{39421263900} & \num{150} & NO \\
\textit{Daucus carota} & \href{https://trace.ncbi.nlm.nih.gov/Traces/sra/?run=SRR7601255}{SRR7601255} & \href{https://www.ncbi.nlm.nih.gov/nuccore/NC_008325}{NC\_008325} & \num{208788541} & \num{63054139382} & \num{151} & Yes \\
\textit{Dendrobium aphyllum} & \href{https://trace.ncbi.nlm.nih.gov/Traces/sra/?run=SRR3932123}{SRR3932123} & \href{https://www.ncbi.nlm.nih.gov/nuccore/NC_035322}{NC\_035322} & \num{24390720} & \num{7317216000} & \num{150} & Yes \\
\textit{Dendrobium chrysanthum} & \href{https://trace.ncbi.nlm.nih.gov/Traces/sra/?run=SRR3932147}{SRR3932147} & \href{https://www.ncbi.nlm.nih.gov/nuccore/NC_035336}{NC\_035336} & \num{12371613} & \num{3711483900} & \num{150} & Yes \\
\textit{Dendrobium chrysotoxum} & \href{https://trace.ncbi.nlm.nih.gov/Traces/sra/?run=SRR3932141}{SRR3932141} & \href{https://www.ncbi.nlm.nih.gov/nuccore/NC_028549}{NC\_028549} & \num{12969805} & \num{3890941500} & \num{150} & Yes \\
\textit{Dendrobium crepidatum} & \href{https://trace.ncbi.nlm.nih.gov/Traces/sra/?run=SRR3932124}{SRR3932124} & \href{https://www.ncbi.nlm.nih.gov/nuccore/NC_035331}{NC\_035331} & \num{10220355} & \num{3066106500} & \num{150} & NO \\
\textit{Dendrobium denneanum} & \href{https://trace.ncbi.nlm.nih.gov/Traces/sra/?run=SRR3932144}{SRR3932144} & \href{https://www.ncbi.nlm.nih.gov/nuccore/NC_035324}{NC\_035324} & \num{11765400} & \num{3529620000} & \num{150} & Yes \\
\textit{Dendrobium falconeri} & \href{https://trace.ncbi.nlm.nih.gov/Traces/sra/?run=SRR3932139}{SRR3932139} & \href{https://www.ncbi.nlm.nih.gov/nuccore/NC_035326}{NC\_035326} & \num{11326112} & \num{3397833600} & \num{150} & NO \\
\textit{Dendrobium nobile} & \href{https://trace.ncbi.nlm.nih.gov/Traces/sra/?run=SRR3932122}{SRR3932122} & \href{https://www.ncbi.nlm.nih.gov/nuccore/NC_029456}{NC\_029456} & \num{11410037} & \num{3423011100} & \num{150} & Yes \\
\textit{Dendrobium parishii} & \href{https://trace.ncbi.nlm.nih.gov/Traces/sra/?run=SRR3932128}{SRR3932128} & \href{https://www.ncbi.nlm.nih.gov/nuccore/NC_035339}{NC\_035339} & \num{14183700} & \num{4255110000} & \num{150} & NO \\
\textit{Dendrobium pendulum} & \href{https://trace.ncbi.nlm.nih.gov/Traces/sra/?run=SRR3932138}{SRR3932138} & \href{https://www.ncbi.nlm.nih.gov/nuccore/NC_029705}{NC\_029705} & \num{11357541} & \num{3407262300} & \num{150} & Yes \\
\textit{Dendrobium primulinum} & \href{https://trace.ncbi.nlm.nih.gov/Traces/sra/?run=SRR3932120}{SRR3932120} & \href{https://www.ncbi.nlm.nih.gov/nuccore/NC_035321}{NC\_035321} & \num{10864205} & \num{3259261500} & \num{150} & NO \\
\textit{Dendrobium wardianum} & \href{https://trace.ncbi.nlm.nih.gov/Traces/sra/?run=SRR3932135}{SRR3932135} & \href{https://www.ncbi.nlm.nih.gov/nuccore/NC_035329}{NC\_035329} & \num{9367500} & \num{2810250000} & \num{150} & Yes \\
\textit{Deschampsia antarctica} & \href{https://trace.ncbi.nlm.nih.gov/Traces/sra/?run=SRR1158316}{SRR1158316} & \href{https://www.ncbi.nlm.nih.gov/nuccore/NC_023533}{NC\_023533} & \num{153346825} & \num{30976058650} & \num{101} & Yes \\
\textit{Digitalis lanata} & \href{https://trace.ncbi.nlm.nih.gov/Traces/sra/?run=SRR5602573}{SRR5602573} & \href{https://www.ncbi.nlm.nih.gov/nuccore/NC_034688}{NC\_034688} & \num{1257524} & \num{730269772} & \num{290} & NO \\
\textit{Dionaea muscipula} & \href{https://trace.ncbi.nlm.nih.gov/Traces/sra/?run=SRR7072324}{SRR7072324} & \href{https://www.ncbi.nlm.nih.gov/nuccore/NC_035417}{NC\_035417} & \num{16215933} & \num{4897211766} & \num{151} & NO \\
\textit{Dioscorea rotundata} & \href{https://trace.ncbi.nlm.nih.gov/Traces/sra/?run=DRR063110}{DRR063110} & \href{https://www.ncbi.nlm.nih.gov/nuccore/NC_024170}{NC\_024170} & \num{18637125} & \num{9295053322} & \num{249} & NO \\
\textit{Dioscorea villosa} & \href{https://trace.ncbi.nlm.nih.gov/Traces/sra/?run=SRR5602590}{SRR5602590} & \href{https://www.ncbi.nlm.nih.gov/nuccore/NC_034686}{NC\_034686} & \num{1447023} & \num{858275557} & \num{297} & Yes \\
\textit{Diospyros lotus} & \href{https://trace.ncbi.nlm.nih.gov/Traces/sra/?run=SRR1560932}{SRR1560932} & \href{https://www.ncbi.nlm.nih.gov/nuccore/NC_030786}{NC\_030786} & \num{8240803} & \num{1648160600} & \num{100} & Yes \\
\textit{Drosera erythrorhiza} & \href{https://trace.ncbi.nlm.nih.gov/Traces/sra/?run=SRR7072766}{SRR7072766} & \href{https://www.ncbi.nlm.nih.gov/nuccore/NC_035241}{NC\_035241} & \num{5832398} & \num{1761384196} & \num{151} & NO \\
\textit{Drosera regia} & \href{https://trace.ncbi.nlm.nih.gov/Traces/sra/?run=SRR7072322}{SRR7072322} & \href{https://www.ncbi.nlm.nih.gov/nuccore/NC_035415}{NC\_035415} & \num{1341304} & \num{405073808} & \num{151} & NO \\
\textit{Dunaliella salina} & \href{https://trace.ncbi.nlm.nih.gov/Traces/sra/?run=SRR3923695}{SRR3923695} & \href{https://www.ncbi.nlm.nih.gov/nuccore/NC_016732}{NC\_016732} & \num{164049728} & \num{49214918400} & \num{150} & Yes \\
\textit{Echinacea angustifolia} & \href{https://trace.ncbi.nlm.nih.gov/Traces/sra/?run=SRR5602579}{SRR5602579} & \href{https://www.ncbi.nlm.nih.gov/nuccore/NC_034324}{NC\_034324} & \num{1669371} & \num{878253310} & \num{263} & Yes \\
\textit{Echinacea atrorubens} & \href{https://trace.ncbi.nlm.nih.gov/Traces/sra/?run=SRR5602578}{SRR5602578} & \href{https://www.ncbi.nlm.nih.gov/nuccore/NC_034323}{NC\_034323} & \num{961923} & \num{472814810} & \num{246} & NO \\
\textit{Echinacea laevigata} & \href{https://trace.ncbi.nlm.nih.gov/Traces/sra/?run=SRR5602581}{SRR5602581} & \href{https://www.ncbi.nlm.nih.gov/nuccore/NC_034322}{NC\_034322} & \num{1099311} & \num{545173423} & \num{248} & NO \\
\textit{Echinacea pallida} & \href{https://trace.ncbi.nlm.nih.gov/Traces/sra/?run=SRR5602580}{SRR5602580} & \href{https://www.ncbi.nlm.nih.gov/nuccore/NC_034321}{NC\_034321} & \num{2039307} & \num{832368111} & \num{204} & Yes \\
\textit{Echinacea paradoxa} & \href{https://trace.ncbi.nlm.nih.gov/Traces/sra/?run=SRR5602575}{SRR5602575} & \href{https://www.ncbi.nlm.nih.gov/nuccore/NC_034320}{NC\_034320} & \num{3101240} & \num{1692835034} & \num{273} & NO \\
\textit{Echinacea purpurea} & \href{https://trace.ncbi.nlm.nih.gov/Traces/sra/?run=SRR5602574}{SRR5602574} & \href{https://www.ncbi.nlm.nih.gov/nuccore/NC_034327}{NC\_034327} & \num{5197414} & \num{2531457606} & \num{244} & Yes \\
\textit{Echinacea sanguinea} & \href{https://trace.ncbi.nlm.nih.gov/Traces/sra/?run=SRR5602577}{SRR5602577} & \href{https://www.ncbi.nlm.nih.gov/nuccore/NC_034328}{NC\_034328} & \num{4911880} & \num{2224965457} & \num{226} & Yes \\
\textit{Echinacea speciosa} & \href{https://trace.ncbi.nlm.nih.gov/Traces/sra/?run=SRR5602576}{SRR5602576} & \href{https://www.ncbi.nlm.nih.gov/nuccore/NC_034325}{NC\_034325} & \num{970715} & \num{483247492} & \num{249} & NO \\
\textit{Echinacea tennesseensis} & \href{https://trace.ncbi.nlm.nih.gov/Traces/sra/?run=SRR5602587}{SRR5602587} & \href{https://www.ncbi.nlm.nih.gov/nuccore/NC_034326}{NC\_034326} & \num{907178} & \num{434759190} & \num{240} & NO \\
\textit{Echinochloa crus-galli} & \href{https://trace.ncbi.nlm.nih.gov/Traces/sra/?run=SRR5920285}{SRR5920285} & \href{https://www.ncbi.nlm.nih.gov/nuccore/NC_028719}{NC\_028719} & \num{203467365} & \num{50866841250} & \num{125} & NO \\
\textit{Echinochloa oryzicola} & \href{https://trace.ncbi.nlm.nih.gov/Traces/sra/?run=SRR5902661}{SRR5902661} & \href{https://www.ncbi.nlm.nih.gov/nuccore/NC_024643}{NC\_024643} & \num{129112810} & \num{32278202500} & \num{125} & Yes \\
\textit{Elaeis guineensis} & \href{https://trace.ncbi.nlm.nih.gov/Traces/sra/?run=ERR276848}{ERR276848} & \href{https://www.ncbi.nlm.nih.gov/nuccore/NC_017602}{NC\_017602} & \num{143340222} & \num{28954724844} & \num{101} & Yes \\
\textit{Eleutherococcus senticosus} & \href{https://trace.ncbi.nlm.nih.gov/Traces/sra/?run=SRR5602586}{SRR5602586} & \href{https://www.ncbi.nlm.nih.gov/nuccore/NC_016430}{NC\_016430} & \num{424930} & \num{211982751} & \num{249} & NO \\
\textit{Epimedium koreanum} & \href{https://trace.ncbi.nlm.nih.gov/Traces/sra/?run=SRR8534306}{SRR8534306} & \href{https://www.ncbi.nlm.nih.gov/nuccore/NC_029943}{NC\_029943} & \num{8308432} & \num{1262881664} & \num{76} & NO \\
\textit{Epimedium sagittatum} & \href{https://trace.ncbi.nlm.nih.gov/Traces/sra/?run=SRR8275049}{SRR8275049} & \href{https://www.ncbi.nlm.nih.gov/nuccore/NC_029428}{NC\_029428} & \num{4141262} & \num{629471824} & \num{76} & NO \\
\textit{Eragrostis tef} & \href{https://trace.ncbi.nlm.nih.gov/Traces/sra/?run=SRR1463402}{SRR1463402} & \href{https://www.ncbi.nlm.nih.gov/nuccore/NC_029413}{NC\_029413} & \num{85023843} & \num{14454053310} & \num{85} & NO \\
\textit{Eriobotrya japonica} & \href{https://trace.ncbi.nlm.nih.gov/Traces/sra/?run=SRR5602604}{SRR5602604} & \href{https://www.ncbi.nlm.nih.gov/nuccore/NC_034639}{NC\_034639} & \num{1852832} & \num{918799928} & \num{248} & Yes \\
\textit{Erodium chrysanthum} & \href{https://trace.ncbi.nlm.nih.gov/Traces/sra/?run=SRR576525}{SRR576525} & \href{https://www.ncbi.nlm.nih.gov/nuccore/NC_027065}{NC\_027065} & \num{31570835} & \num{6314167000} & \num{100} & NO \\
\textit{Erodium texanum} & \href{https://trace.ncbi.nlm.nih.gov/Traces/sra/?run=ERR2799526}{ERR2799526} & \href{https://www.ncbi.nlm.nih.gov/nuccore/NC_014569}{NC\_014569} & \num{1158815} & \num{349962130} & \num{151} & Yes \\
\textit{Erythranthe lutea} & \href{https://trace.ncbi.nlm.nih.gov/Traces/sra/?run=SRR5307620}{SRR5307620} & \href{https://www.ncbi.nlm.nih.gov/nuccore/NC_030212}{NC\_030212} & \num{10000000} & \num{2000000000} & \num{100} & Yes \\
\textit{Eucalyptus grandis} & \href{https://trace.ncbi.nlm.nih.gov/Traces/sra/?run=SRR1392700}{SRR1392700} & \href{https://www.ncbi.nlm.nih.gov/nuccore/NC_014570}{NC\_014570} & \num{327608152} & \num{98282445600} & \num{150} & Yes \\
\textit{Euphorbia esula} & \href{https://trace.ncbi.nlm.nih.gov/Traces/sra/?run=SRR6355713}{SRR6355713} & \href{https://www.ncbi.nlm.nih.gov/nuccore/NC_033910}{NC\_033910} & \num{65533012} & \num{13106602400} & \num{100} & NO \\
\textit{Fagopyrum tataricum} & \href{https://trace.ncbi.nlm.nih.gov/Traces/sra/?run=SRR5433722}{SRR5433722} & \href{https://www.ncbi.nlm.nih.gov/nuccore/NC_027161}{NC\_027161} & \num{15222050} & \num{6981058500} & \num{229} & Yes \\
\textit{Festuca arundinacea} & \href{https://trace.ncbi.nlm.nih.gov/Traces/sra/?run=SRR6885354}{SRR6885354} & \href{https://www.ncbi.nlm.nih.gov/nuccore/NC_011713}{NC\_011713} & \num{6197329} & \num{3085916062} & \num{249} & NO \\
\textit{Ficus carica} & \href{https://trace.ncbi.nlm.nih.gov/Traces/sra/?run=SRR5678803}{SRR5678803} & \href{https://www.ncbi.nlm.nih.gov/nuccore/NC_035237}{NC\_035237} & \num{26827466} & \num{15073339221} & \num{281} & NO \\
\textit{Ficus racemosa} & \href{https://trace.ncbi.nlm.nih.gov/Traces/sra/?run=SRR1405699}{SRR1405699} & \href{https://www.ncbi.nlm.nih.gov/nuccore/NC_028185}{NC\_028185} & \num{31025867} & \num{4715931784} & \num{76} & NO \\
\textit{Foeniculum vulgare} & \href{https://trace.ncbi.nlm.nih.gov/Traces/sra/?run=SRR8690417}{SRR8690417} & \href{https://www.ncbi.nlm.nih.gov/nuccore/NC_029469}{NC\_029469} & \num{3715298} & \num{750490196} & \num{101} & NO \\
\textit{Fragaria chiloensis} & \href{https://trace.ncbi.nlm.nih.gov/Traces/sra/?run=SRR1612828}{SRR1612828} & \href{https://www.ncbi.nlm.nih.gov/nuccore/NC_019601}{NC\_019601} & \num{43583617} & \num{10460068080} & \num{120} & NO \\
\textit{Fragaria vesca subsp. bracteata} & \href{https://trace.ncbi.nlm.nih.gov/Traces/sra/?run=SRR866156}{SRR866156} & \href{https://www.ncbi.nlm.nih.gov/nuccore/NC_018766}{NC\_018766} & \num{4311362} & \num{819158780} & \num{95} & Yes \\
\textit{Fragaria virginiana} & \href{https://trace.ncbi.nlm.nih.gov/Traces/sra/?run=SRR5602605}{SRR5602605} & \href{https://www.ncbi.nlm.nih.gov/nuccore/NC_019602}{NC\_019602} & \num{1199457} & \num{708866398} & \num{295} & Yes \\
\textit{Francoa sonchifolia} & \href{https://trace.ncbi.nlm.nih.gov/Traces/sra/?run=SRR576529}{SRR576529} & \href{https://www.ncbi.nlm.nih.gov/nuccore/NC_021101}{NC\_021101} & \num{32806903} & \num{6561380600} & \num{100} & NO \\
\textit{Geranium maderense} & \href{https://trace.ncbi.nlm.nih.gov/Traces/sra/?run=ERR2799530}{ERR2799530} & \href{https://www.ncbi.nlm.nih.gov/nuccore/NC_029999}{NC\_029999} & \num{1031538} & \num{311524476} & \num{151} & Yes \\
\textit{Ginkgo biloba} & \href{https://trace.ncbi.nlm.nih.gov/Traces/sra/?run=SRR3089810}{SRR3089810} & \href{https://www.ncbi.nlm.nih.gov/nuccore/NC_016986}{NC\_016986} & \num{10164078} & \num{2541019500} & \num{125} & NO \\
\textit{Glycine dolichocarpa} & \href{https://trace.ncbi.nlm.nih.gov/Traces/sra/?run=SRR1174380}{SRR1174380} & \href{https://www.ncbi.nlm.nih.gov/nuccore/NC_021648}{NC\_021648} & \num{44850853} & \num{9059872306} & \num{101} & NO \\
\textit{Glycine max} & \href{https://trace.ncbi.nlm.nih.gov/Traces/sra/?run=SRR6796784}{SRR6796784} & \href{https://www.ncbi.nlm.nih.gov/nuccore/NC_007942}{NC\_007942} & \num{70339371} & \num{21242490042} & \num{151} & NO \\
\textit{Glycine soja} & \href{https://trace.ncbi.nlm.nih.gov/Traces/sra/?run=SRR7725010}{SRR7725010} & \href{https://www.ncbi.nlm.nih.gov/nuccore/NC_022868}{NC\_022868} & \num{131827844} & \num{39548353200} & \num{150} & Yes \\
\textit{Glycine syndetika} & \href{https://trace.ncbi.nlm.nih.gov/Traces/sra/?run=SRR1176843}{SRR1176843} & \href{https://www.ncbi.nlm.nih.gov/nuccore/NC_021650}{NC\_021650} & \num{51528078} & \num{10408671756} & \num{101} & NO \\
\textit{Glycine tomentella} & \href{https://trace.ncbi.nlm.nih.gov/Traces/sra/?run=SRR6823488}{SRR6823488} & \href{https://www.ncbi.nlm.nih.gov/nuccore/NC_021636}{NC\_021636} & \num{265989287} & \num{131814519905} & \num{248} & NO \\
\textit{Glycyrrhiza glabra} & \href{https://trace.ncbi.nlm.nih.gov/Traces/sra/?run=SRR8690419}{SRR8690419} & \href{https://www.ncbi.nlm.nih.gov/nuccore/NC_024038}{NC\_024038} & \num{4940810} & \num{998043620} & \num{101} & Yes \\
\textit{Glyptostrobus pensilis} & \href{https://trace.ncbi.nlm.nih.gov/Traces/sra/?run=ERR2799543}{ERR2799543} & \href{https://www.ncbi.nlm.nih.gov/nuccore/NC_031354}{NC\_031354} & \num{1061851} & \num{320679002} & \num{151} & NO \\
\textit{Gnetum gnemon} & \href{https://trace.ncbi.nlm.nih.gov/Traces/sra/?run=ERR268420}{ERR268420} & \href{https://www.ncbi.nlm.nih.gov/nuccore/NC_026301}{NC\_026301} & \num{146146748} & \num{29229349600} & \num{100} & NO \\
\textit{Gossypium anomalum} & \href{https://trace.ncbi.nlm.nih.gov/Traces/sra/?run=SRR3560153}{SRR3560153} & \href{https://www.ncbi.nlm.nih.gov/nuccore/NC_023213}{NC\_023213} & \num{152611035} & \num{30827429070} & \num{101} & NO \\
\textit{Gossypium arboreum} & \href{https://trace.ncbi.nlm.nih.gov/Traces/sra/?run=SRR2012734}{SRR2012734} & \href{https://www.ncbi.nlm.nih.gov/nuccore/NC_016712}{NC\_016712} & \num{8946554} & \num{5385825508} & \num{301} & NO \\
\textit{Gossypium aridum} & \href{https://trace.ncbi.nlm.nih.gov/Traces/sra/?run=SRR8136263}{SRR8136263} & \href{https://www.ncbi.nlm.nih.gov/nuccore/NC_033396}{NC\_033396} & \num{158194517} & \num{31638903400} & \num{100} & Yes \\
\textit{Gossypium armourianum} & \href{https://trace.ncbi.nlm.nih.gov/Traces/sra/?run=SRR8136258}{SRR8136258} & \href{https://www.ncbi.nlm.nih.gov/nuccore/NC_033400}{NC\_033400} & \num{155005858} & \num{31001171600} & \num{100} & NO \\
\textit{Gossypium barbadense} & \href{https://trace.ncbi.nlm.nih.gov/Traces/sra/?run=SRR8627357}{SRR8627357} & \href{https://www.ncbi.nlm.nih.gov/nuccore/NC_008641}{NC\_008641} & \num{159103409} & \num{79869911318} & \num{251} & NO \\
\textit{Gossypium bickii} & \href{https://trace.ncbi.nlm.nih.gov/Traces/sra/?run=SRR3560187}{SRR3560187} & \href{https://www.ncbi.nlm.nih.gov/nuccore/NC_023214}{NC\_023214} & \num{179898006} & \num{35979601200} & \num{100} & Yes \\
\textit{Gossypium darwinii} & \href{https://trace.ncbi.nlm.nih.gov/Traces/sra/?run=SRR8545538}{SRR8545538} & \href{https://www.ncbi.nlm.nih.gov/nuccore/NC_016670}{NC\_016670} & \num{440216228} & \num{132945300856} & \num{151} & NO \\
\textit{Gossypium davidsonii} & \href{https://trace.ncbi.nlm.nih.gov/Traces/sra/?run=SRR6334584}{SRR6334584} & \href{https://www.ncbi.nlm.nih.gov/nuccore/NC_033395}{NC\_033395} & \num{117719642} & \num{35315892600} & \num{150} & NO \\
\textit{Gossypium gossypioides} & \href{https://trace.ncbi.nlm.nih.gov/Traces/sra/?run=SRR3560148}{SRR3560148} & \href{https://www.ncbi.nlm.nih.gov/nuccore/NC_017894}{NC\_017894} & \num{153288210} & \num{30657642000} & \num{100} & NO \\
\textit{Gossypium harknessii} & \href{https://trace.ncbi.nlm.nih.gov/Traces/sra/?run=SRR8136257}{SRR8136257} & \href{https://www.ncbi.nlm.nih.gov/nuccore/NC_033333}{NC\_033333} & \num{183115691} & \num{36431624874} & \num{99} & NO \\
\textit{Gossypium herbaceum} & \href{https://trace.ncbi.nlm.nih.gov/Traces/sra/?run=SRR2012721}{SRR2012721} & \href{https://www.ncbi.nlm.nih.gov/nuccore/NC_023215}{NC\_023215} & \num{12440256} & \num{7489034112} & \num{301} & NO \\
\textit{Gossypium hirsutum} & \href{https://trace.ncbi.nlm.nih.gov/Traces/sra/?run=SRR3560152}{SRR3560152} & \href{https://www.ncbi.nlm.nih.gov/nuccore/NC_007944}{NC\_007944} & \num{4487735} & \num{2205113705} & \num{246} & Yes \\
\textit{Gossypium klotzschianum} & \href{https://trace.ncbi.nlm.nih.gov/Traces/sra/?run=SRR8136264}{SRR8136264} & \href{https://www.ncbi.nlm.nih.gov/nuccore/NC_033394}{NC\_033394} & \num{161773353} & \num{32354670600} & \num{100} & NO \\
\textit{Gossypium longicalyx} & \href{https://trace.ncbi.nlm.nih.gov/Traces/sra/?run=SRR617704}{SRR617704} & \href{https://www.ncbi.nlm.nih.gov/nuccore/NC_023216}{NC\_023216} & \num{534258839} & \num{107920285478} & \num{101} & Yes \\
\textit{Gossypium mustelinum} & \href{https://trace.ncbi.nlm.nih.gov/Traces/sra/?run=SRR8727328}{SRR8727328} & \href{https://www.ncbi.nlm.nih.gov/nuccore/NC_016711}{NC\_016711} & \num{158115596} & \num{79374029192} & \num{251} & NO \\
\textit{Gossypium raimondii} & \href{https://trace.ncbi.nlm.nih.gov/Traces/sra/?run=ERR1449077}{ERR1449077} & \href{https://www.ncbi.nlm.nih.gov/nuccore/NC_016668}{NC\_016668} & \num{21720967} & \num{6559732034} & \num{151} & NO \\
\textit{Gossypium somalense} & \href{https://trace.ncbi.nlm.nih.gov/Traces/sra/?run=SRR3560160}{SRR3560160} & \href{https://www.ncbi.nlm.nih.gov/nuccore/NC_018110}{NC\_018110} & \num{69625198} & \num{13925039600} & \num{100} & Yes \\
\textit{Gossypium sturtianum} & \href{https://trace.ncbi.nlm.nih.gov/Traces/sra/?run=SRR3560184}{SRR3560184} & \href{https://www.ncbi.nlm.nih.gov/nuccore/NC_023218}{NC\_023218} & \num{95466009} & \num{19093201800} & \num{100} & NO \\
\textit{Gossypium thurberi} & \href{https://trace.ncbi.nlm.nih.gov/Traces/sra/?run=SRR8076131}{SRR8076131} & \href{https://www.ncbi.nlm.nih.gov/nuccore/NC_015204}{NC\_015204} & \num{181003659} & \num{36200731800} & \num{100} & Yes \\
\textit{Gossypium tomentosum} & \href{https://trace.ncbi.nlm.nih.gov/Traces/sra/?run=SRR6334536}{SRR6334536} & \href{https://www.ncbi.nlm.nih.gov/nuccore/NC_016690}{NC\_016690} & \num{138849836} & \num{41654950800} & \num{150} & NO \\
\textit{Gossypium trilobum} & \href{https://trace.ncbi.nlm.nih.gov/Traces/sra/?run=SRR3560157}{SRR3560157} & \href{https://www.ncbi.nlm.nih.gov/nuccore/NC_033397}{NC\_033397} & \num{52816676} & \num{10563335200} & \num{100} & NO \\
\textit{Gossypium turneri} & \href{https://trace.ncbi.nlm.nih.gov/Traces/sra/?run=SRR8136254}{SRR8136254} & \href{https://www.ncbi.nlm.nih.gov/nuccore/NC_026835}{NC\_026835} & \num{124131681} & \num{37239504300} & \num{150} & NO \\
\textit{Haberlea rhodopensis} & \href{https://trace.ncbi.nlm.nih.gov/Traces/sra/?run=SRR4428742}{SRR4428742} & \href{https://www.ncbi.nlm.nih.gov/nuccore/NC_031852}{NC\_031852} & \num{8365536} & \num{1673107200} & \num{100} & NO \\
\textit{Haplostachys haplostachya} & \href{https://trace.ncbi.nlm.nih.gov/Traces/sra/?run=SRR3170741}{SRR3170741} & \href{https://www.ncbi.nlm.nih.gov/nuccore/NC_029819}{NC\_029819} & \num{251836} & \num{76054472} & \num{151} & NO \\
\textit{Helianthus annuus} & \href{https://trace.ncbi.nlm.nih.gov/Traces/sra/?run=SRR825830}{SRR825830} & \href{https://www.ncbi.nlm.nih.gov/nuccore/NC_007977}{NC\_007977} & \num{57562191} & \num{13124179548} & \num{114} & Yes \\
\textit{Helianthus argophyllus} & \href{https://trace.ncbi.nlm.nih.gov/Traces/sra/?run=SRR2155086}{SRR2155086} & \href{https://www.ncbi.nlm.nih.gov/nuccore/NC_030275}{NC\_030275} & \num{12784471} & \num{2582463142} & \num{101} & NO \\
\textit{Helianthus debilis} & \href{https://trace.ncbi.nlm.nih.gov/Traces/sra/?run=SRR5907791}{SRR5907791} & \href{https://www.ncbi.nlm.nih.gov/nuccore/NC_030173}{NC\_030173} & \num{7323836} & \num{1464767200} & \num{100} & NO \\
\textit{Helianthus divaricatus} & \href{https://trace.ncbi.nlm.nih.gov/Traces/sra/?run=SRR3492253}{SRR3492253} & \href{https://www.ncbi.nlm.nih.gov/nuccore/NC_023109}{NC\_023109} & \num{13246221} & \num{2649244200} & \num{100} & NO \\
\textit{Helianthus giganteus} & \href{https://trace.ncbi.nlm.nih.gov/Traces/sra/?run=SRR5907716}{SRR5907716} & \href{https://www.ncbi.nlm.nih.gov/nuccore/NC_023107}{NC\_023107} & \num{20626828} & \num{4125365600} & \num{100} & NO \\
\textit{Helianthus grosseserratus} & \href{https://trace.ncbi.nlm.nih.gov/Traces/sra/?run=SRR5907715}{SRR5907715} & \href{https://www.ncbi.nlm.nih.gov/nuccore/NC_023108}{NC\_023108} & \num{23480167} & \num{4696033400} & \num{100} & Yes \\
\textit{Heteropogon triticeus} & \href{https://trace.ncbi.nlm.nih.gov/Traces/sra/?run=SRR5446059}{SRR5446059} & \href{https://www.ncbi.nlm.nih.gov/nuccore/NC_035025}{NC\_035025} & \num{14302809} & \num{4290842700} & \num{150} & Yes \\
\textit{Hibiscus syriacus} & \href{https://trace.ncbi.nlm.nih.gov/Traces/sra/?run=SRR1265942}{SRR1265942} & \href{https://www.ncbi.nlm.nih.gov/nuccore/NC_026909}{NC\_026909} & \num{25909841} & \num{5233787882} & \num{101} & NO \\
\textit{Hirtella physophora} & \href{https://trace.ncbi.nlm.nih.gov/Traces/sra/?run=SRR1179646}{SRR1179646} & \href{https://www.ncbi.nlm.nih.gov/nuccore/NC_024066}{NC\_024066} & \num{18840660} & \num{3768132000} & \num{100} & Yes \\
\textit{Hirtella racemosa} & \href{https://trace.ncbi.nlm.nih.gov/Traces/sra/?run=SRR1179649}{SRR1179649} & \href{https://www.ncbi.nlm.nih.gov/nuccore/NC_024060}{NC\_024060} & \num{8044385} & \num{1624965770} & \num{101} & Yes \\
\textit{Hordeum vulgare subsp. vulgare} & \href{https://trace.ncbi.nlm.nih.gov/Traces/sra/?run=SRR490932}{SRR490932} & \href{https://www.ncbi.nlm.nih.gov/nuccore/NC_008590}{NC\_008590} & \num{262845212} & \num{65711303000} & \num{125} & NO \\
\textit{Humulus lupulus} & \href{https://trace.ncbi.nlm.nih.gov/Traces/sra/?run=SRR8690418}{SRR8690418} & \href{https://www.ncbi.nlm.nih.gov/nuccore/NC_028032}{NC\_028032} & \num{6541057} & \num{1321293514} & \num{101} & Yes \\
\textit{Hydrastis canadensis} & \href{https://trace.ncbi.nlm.nih.gov/Traces/sra/?run=SRR5602606}{SRR5602606} & \href{https://www.ncbi.nlm.nih.gov/nuccore/NC_034702}{NC\_034702} & \num{1356811} & \num{671823274} & \num{248} & Yes \\
\textit{Hyoscyamus niger} & \href{https://trace.ncbi.nlm.nih.gov/Traces/sra/?run=SRR1508958}{SRR1508958} & \href{https://www.ncbi.nlm.nih.gov/nuccore/NC_024261}{NC\_024261} & \num{21000000} & \num{4200000000} & \num{100} & Yes \\
\textit{Hypseocharis bilobata} & \href{https://trace.ncbi.nlm.nih.gov/Traces/sra/?run=SRR576531}{SRR576531} & \href{https://www.ncbi.nlm.nih.gov/nuccore/NC_023260}{NC\_023260} & \num{32280852} & \num{6456170400} & \num{100} & NO \\
\textit{Illicium anisatum} & \href{https://trace.ncbi.nlm.nih.gov/Traces/sra/?run=SRR5602608}{SRR5602608} & \href{https://www.ncbi.nlm.nih.gov/nuccore/NC_034703}{NC\_034703} & \num{3195974} & \num{961644484} & \num{150} & NO \\
\textit{Illicium floridanum} & \href{https://trace.ncbi.nlm.nih.gov/Traces/sra/?run=SRR5602609}{SRR5602609} & \href{https://www.ncbi.nlm.nih.gov/nuccore/NC_034685}{NC\_034685} & \num{1929116} & \num{1141799529} & \num{296} & NO \\
\textit{Illicium henryi} & \href{https://trace.ncbi.nlm.nih.gov/Traces/sra/?run=SRR5602610}{SRR5602610} & \href{https://www.ncbi.nlm.nih.gov/nuccore/NC_034699}{NC\_034699} & \num{1240196} & \num{611286832} & \num{246} & NO \\
\textit{Illicium verum} & \href{https://trace.ncbi.nlm.nih.gov/Traces/sra/?run=SRR5602607}{SRR5602607} & \href{https://www.ncbi.nlm.nih.gov/nuccore/NC_034689}{NC\_034689} & \num{2752799} & \num{828427309} & \num{150} & NO \\
\textit{Ipomoea batatas} & \href{https://trace.ncbi.nlm.nih.gov/Traces/sra/?run=SRR7868503}{SRR7868503} & \href{https://www.ncbi.nlm.nih.gov/nuccore/NC_026703}{NC\_026703} & \num{55433003} & \num{27827367506} & \num{251} & NO \\
\textit{Ipomoea nil} & \href{https://trace.ncbi.nlm.nih.gov/Traces/sra/?run=DRR013917}{DRR013917} & \href{https://www.ncbi.nlm.nih.gov/nuccore/NC_031159}{NC\_031159} & \num{149492354} & \num{44847706200} & \num{150} & NO \\
\textit{Ipomoea trifida} & \href{https://trace.ncbi.nlm.nih.gov/Traces/sra/?run=SRR6667669}{SRR6667669} & \href{https://www.ncbi.nlm.nih.gov/nuccore/NC_034670}{NC\_034670} & \num{148892802} & \num{47645696640} & \num{160} & NO \\
\textit{Jasminum sambac} & \href{https://trace.ncbi.nlm.nih.gov/Traces/sra/?run=SRR5602611}{SRR5602611} & \href{https://www.ncbi.nlm.nih.gov/nuccore/NC_034694}{NC\_034694} & \num{4487750} & \num{2661867968} & \num{297} & NO \\
\textit{Jasminum tortuosum} & \href{https://trace.ncbi.nlm.nih.gov/Traces/sra/?run=SRR5602601}{SRR5602601} & \href{https://www.ncbi.nlm.nih.gov/nuccore/NC_034691}{NC\_034691} & \num{1468745} & \num{726257599} & \num{247} & NO \\
\textit{Jatropha curcas} & \href{https://trace.ncbi.nlm.nih.gov/Traces/sra/?run=DRR001794}{DRR001794} & \href{https://www.ncbi.nlm.nih.gov/nuccore/NC_012224}{NC\_012224} & \num{99881947} & \num{15182055944} & \num{76} & NO \\
\textit{Juglans regia} & \href{https://trace.ncbi.nlm.nih.gov/Traces/sra/?run=SRR2057822}{SRR2057822} & \href{https://www.ncbi.nlm.nih.gov/nuccore/NC_028617}{NC\_028617} & \num{111991280} & \num{33821366560} & \num{151} & Yes \\
\textit{Juniperus cedrus} & \href{https://trace.ncbi.nlm.nih.gov/Traces/sra/?run=SRR1145775}{SRR1145775} & \href{https://www.ncbi.nlm.nih.gov/nuccore/NC_028190}{NC\_028190} & \num{46193335} & \num{9238667000} & \num{100} & Yes \\
\textit{Juniperus communis} & \href{https://trace.ncbi.nlm.nih.gov/Traces/sra/?run=ERR268423}{ERR268423} & \href{https://www.ncbi.nlm.nih.gov/nuccore/NC_035068}{NC\_035068} & \num{472881723} & \num{94576344600} & \num{100} & NO \\
\textit{Lactuca sativa} & \href{https://trace.ncbi.nlm.nih.gov/Traces/sra/?run=SRR8736654}{SRR8736654} & \href{https://www.ncbi.nlm.nih.gov/nuccore/NC_007578}{NC\_007578} & \num{267081494} & \num{80124448200} & \num{150} & Yes \\
\textit{Lathyrus sativus} & \href{https://trace.ncbi.nlm.nih.gov/Traces/sra/?run=ERR413118}{ERR413118} & \href{https://www.ncbi.nlm.nih.gov/nuccore/NC_014063}{NC\_014063} & \num{3876258} & \num{783004116} & \num{101} & NO \\
\textit{Laurus nobilis} & \href{https://trace.ncbi.nlm.nih.gov/Traces/sra/?run=SRR5602602}{SRR5602602} & \href{https://www.ncbi.nlm.nih.gov/nuccore/NC_034700}{NC\_034700} & \num{1774932} & \num{880729409} & \num{248} & Yes \\
\textit{Lavandula angustifolia} & \href{https://trace.ncbi.nlm.nih.gov/Traces/sra/?run=SRR5757713}{SRR5757713} & \href{https://www.ncbi.nlm.nih.gov/nuccore/NC_029370}{NC\_029370} & \num{37808338} & \num{11418118076} & \num{151} & NO \\
\textit{Lens culinaris} & \href{https://trace.ncbi.nlm.nih.gov/Traces/sra/?run=ERR413115}{ERR413115} & \href{https://www.ncbi.nlm.nih.gov/nuccore/NC_027152}{NC\_027152} & \num{6895415} & \num{1392873830} & \num{101} & Yes \\
\textit{Licania alba} & \href{https://trace.ncbi.nlm.nih.gov/Traces/sra/?run=SRR1179652}{SRR1179652} & \href{https://www.ncbi.nlm.nih.gov/nuccore/NC_024064}{NC\_024064} & \num{6960753} & \num{1406072106} & \num{101} & Yes \\
\textit{Licania sprucei} & \href{https://trace.ncbi.nlm.nih.gov/Traces/sra/?run=SRR1179651}{SRR1179651} & \href{https://www.ncbi.nlm.nih.gov/nuccore/NC_024065}{NC\_024065} & \num{6492846} & \num{1311554892} & \num{101} & NO \\
\textit{Lilium tsingtauense} & \href{https://trace.ncbi.nlm.nih.gov/Traces/sra/?run=SRR1265940}{SRR1265940} & \href{https://www.ncbi.nlm.nih.gov/nuccore/NC_027675}{NC\_027675} & \num{9313992} & \num{1881426384} & \num{101} & NO \\
\textit{Liriodendron tulipifera} & \href{https://trace.ncbi.nlm.nih.gov/Traces/sra/?run=SRR4240953}{SRR4240953} & \href{https://www.ncbi.nlm.nih.gov/nuccore/NC_008326}{NC\_008326} & \num{4000000} & \num{1208000000} & \num{151} & Yes \\
\textit{Litchi chinensis} & \href{https://trace.ncbi.nlm.nih.gov/Traces/sra/?run=SRR5812499}{SRR5812499} & \href{https://www.ncbi.nlm.nih.gov/nuccore/NC_035238}{NC\_035238} & \num{6720070} & \num{2029461140} & \num{151} & NO \\
\textit{Lotus japonicus} & \href{https://trace.ncbi.nlm.nih.gov/Traces/sra/?run=SRR8115522}{SRR8115522} & \href{https://www.ncbi.nlm.nih.gov/nuccore/NC_002694}{NC\_002694} & \num{175495827} & \num{52999739754} & \num{151} & Yes \\
\textit{Ludisia discolor} & \href{https://trace.ncbi.nlm.nih.gov/Traces/sra/?run=SRR3484539}{SRR3484539} & \href{https://www.ncbi.nlm.nih.gov/nuccore/NC_030540}{NC\_030540} & \num{14073966} & \num{8472527532} & \num{301} & Yes \\
\textit{Macadamia integrifolia} & \href{https://trace.ncbi.nlm.nih.gov/Traces/sra/?run=ERR1397263}{ERR1397263} & \href{https://www.ncbi.nlm.nih.gov/nuccore/NC_025288}{NC\_025288} & \num{28756805} & \num{8684555110} & \num{151} & Yes \\
\textit{Magnolia biondii} & \href{https://trace.ncbi.nlm.nih.gov/Traces/sra/?run=SRR5602588}{SRR5602588} & \href{https://www.ncbi.nlm.nih.gov/nuccore/NC_034687}{NC\_034687} & \num{1600124} & \num{953606691} & \num{298} & NO \\
\textit{Magnolia denudata} & \href{https://trace.ncbi.nlm.nih.gov/Traces/sra/?run=SRR5602603}{SRR5602603} & \href{https://www.ncbi.nlm.nih.gov/nuccore/NC_018357}{NC\_018357} & \num{1640979} & \num{978183192} & \num{298} & Yes \\
\textit{Magnolia officinalis subsp. biloba} & \href{https://trace.ncbi.nlm.nih.gov/Traces/sra/?run=SRR5602589}{SRR5602589} & \href{https://www.ncbi.nlm.nih.gov/nuccore/NC_020317}{NC\_020317} & \num{1744003} & \num{1040030660} & \num{298} & NO \\
\textit{Malus prunifolia} & \href{https://trace.ncbi.nlm.nih.gov/Traces/sra/?run=SRR3571175}{SRR3571175} & \href{https://www.ncbi.nlm.nih.gov/nuccore/NC_031163}{NC\_031163} & \num{22532714} & \num{6804879628} & \num{151} & NO \\
\textit{Mangifera indica} & \href{https://trace.ncbi.nlm.nih.gov/Traces/sra/?run=SRR5812496}{SRR5812496} & \href{https://www.ncbi.nlm.nih.gov/nuccore/NC_035239}{NC\_035239} & \num{9989213} & \num{3016742326} & \num{151} & Yes \\
\textit{Manihot esculenta} & \href{https://trace.ncbi.nlm.nih.gov/Traces/sra/?run=SRR2847384}{SRR2847384} & \href{https://www.ncbi.nlm.nih.gov/nuccore/NC_010433}{NC\_010433} & \num{121616638} & \num{54727487100} & \num{225} & Yes \\
\textit{Mankyua chejuensis} & \href{https://trace.ncbi.nlm.nih.gov/Traces/sra/?run=SRR7630500}{SRR7630500} & \href{https://www.ncbi.nlm.nih.gov/nuccore/NC_017006}{NC\_017006} & \num{7279904} & \num{4382502208} & \num{301} & Yes \\
\textit{Marchantia polymorpha} & \href{https://trace.ncbi.nlm.nih.gov/Traces/sra/?run=SRR396657}{SRR396657} & \href{https://www.ncbi.nlm.nih.gov/nuccore/NC_001319}{NC\_001319} & \num{123055214} & \num{36916564200} & \num{150} & NO \\
\textit{Medicago truncatula} & \href{https://trace.ncbi.nlm.nih.gov/Traces/sra/?run=SRR1028837}{SRR1028837} & \href{https://www.ncbi.nlm.nih.gov/nuccore/NC_003119}{NC\_003119} & \num{29663436} & \num{8958357672} & \num{151} & NO \\
\textit{Melianthus villosus} & \href{https://trace.ncbi.nlm.nih.gov/Traces/sra/?run=SRR576532}{SRR576532} & \href{https://www.ncbi.nlm.nih.gov/nuccore/NC_023256}{NC\_023256} & \num{31695892} & \num{6339178400} & \num{100} & Yes \\
\textit{Mentha longifolia} & \href{https://trace.ncbi.nlm.nih.gov/Traces/sra/?run=SRR3204556}{SRR3204556} & \href{https://www.ncbi.nlm.nih.gov/nuccore/NC_032054}{NC\_032054} & \num{67215785} & \num{13577588570} & \num{101} & Yes \\
\textit{Miscanthus sacchariflorus} & \href{https://trace.ncbi.nlm.nih.gov/Traces/sra/?run=SRR559245}{SRR559245} & \href{https://www.ncbi.nlm.nih.gov/nuccore/NC_028720}{NC\_028720} & \num{73073855} & \num{22068304210} & \num{151} & NO \\
\textit{Miscanthus sinensis} & \href{https://trace.ncbi.nlm.nih.gov/Traces/sra/?run=SRR4028761}{SRR4028761} & \href{https://www.ncbi.nlm.nih.gov/nuccore/NC_028721}{NC\_028721} & \num{34003586} & \num{10269082972} & \num{151} & Yes \\
\textit{Mitragyna speciosa} & \href{https://trace.ncbi.nlm.nih.gov/Traces/sra/?run=SRR5602600}{SRR5602600} & \href{https://www.ncbi.nlm.nih.gov/nuccore/NC_034698}{NC\_034698} & \num{1327534} & \num{658729522} & \num{248} & NO \\
\textit{Monsonia emarginata} & \href{https://trace.ncbi.nlm.nih.gov/Traces/sra/?run=ERR2799532}{ERR2799532} & \href{https://www.ncbi.nlm.nih.gov/nuccore/NC_029694}{NC\_029694} & \num{1089971} & \num{329171242} & \num{151} & Yes \\
\textit{Nelumbo lutea} & \href{https://trace.ncbi.nlm.nih.gov/Traces/sra/?run=SRR5115287}{SRR5115287} & \href{https://www.ncbi.nlm.nih.gov/nuccore/NC_015605}{NC\_015605} & \num{45729480} & \num{11432370000} & \num{125} & Yes \\
\textit{Nelumbo nucifera} & \href{https://trace.ncbi.nlm.nih.gov/Traces/sra/?run=SRR7159815}{SRR7159815} & \href{https://www.ncbi.nlm.nih.gov/nuccore/NC_025339}{NC\_025339} & \num{30517828} & \num{9155348400} & \num{150} & NO \\
\textit{Nicotiana otophora} & \href{https://trace.ncbi.nlm.nih.gov/Traces/sra/?run=SRR1171700}{SRR1171700} & \href{https://www.ncbi.nlm.nih.gov/nuccore/NC_032724}{NC\_032724} & \num{67460219} & \num{13626964238} & \num{101} & NO \\
\textit{Nicotiana tabacum} & \href{https://trace.ncbi.nlm.nih.gov/Traces/sra/?run=SRR954964}{SRR954964} & \href{https://www.ncbi.nlm.nih.gov/nuccore/NC_001879}{NC\_001879} & \num{792406088} & \num{160066029776} & \num{101} & Yes \\
\textit{Nicotiana undulata} & \href{https://trace.ncbi.nlm.nih.gov/Traces/sra/?run=SRR8173251}{SRR8173251} & \href{https://www.ncbi.nlm.nih.gov/nuccore/NC_016068}{NC\_016068} & \num{223212812} & \num{45088988024} & \num{101} & Yes \\
\textit{Ocimum basilicum} & \href{https://trace.ncbi.nlm.nih.gov/Traces/sra/?run=SRR6940087}{SRR6940087} & \href{https://www.ncbi.nlm.nih.gov/nuccore/NC_035143}{NC\_035143} & \num{32031837} & \num{9609551100} & \num{150} & NO \\
\textit{Oenothera biennis} & \href{https://trace.ncbi.nlm.nih.gov/Traces/sra/?run=SRR1771524}{SRR1771524} & \href{https://www.ncbi.nlm.nih.gov/nuccore/NC_010361}{NC\_010361} & \num{13426811} & \num{2658508578} & \num{99} & Yes \\
\textit{Oenothera villaricae} & \href{https://trace.ncbi.nlm.nih.gov/Traces/sra/?run=SRR1771538}{SRR1771538} & \href{https://www.ncbi.nlm.nih.gov/nuccore/NC_030532}{NC\_030532} & \num{3055389} & \num{611077800} & \num{100} & NO \\
\textit{Olea europaea subsp. europaea} & \href{https://trace.ncbi.nlm.nih.gov/Traces/sra/?run=ERR375848}{ERR375848} & \href{https://www.ncbi.nlm.nih.gov/nuccore/NC_015401}{NC\_015401} & \num{27590779} & \num{16729938945} & \num{303} & NO \\
\textit{Oryza australiensis} & \href{https://trace.ncbi.nlm.nih.gov/Traces/sra/?run=DRR056778}{DRR056778} & \href{https://www.ncbi.nlm.nih.gov/nuccore/NC_024608}{NC\_024608} & \num{69350090} & \num{20805027000} & \num{150} & NO \\
\textit{Oryza barthii} & \href{https://trace.ncbi.nlm.nih.gov/Traces/sra/?run=SRR7341625}{SRR7341625} & \href{https://www.ncbi.nlm.nih.gov/nuccore/NC_027460}{NC\_027460} & \num{55484960} & \num{27742480000} & \num{250} & Yes \\
\textit{Oryza brachyantha} & \href{https://trace.ncbi.nlm.nih.gov/Traces/sra/?run=DRR053294}{DRR053294} & \href{https://www.ncbi.nlm.nih.gov/nuccore/NC_030596}{NC\_030596} & \num{10715682} & \num{3214704600} & \num{150} & Yes \\
\textit{Oryza glaberrima} & \href{https://trace.ncbi.nlm.nih.gov/Traces/sra/?run=SRR7341626}{SRR7341626} & \href{https://www.ncbi.nlm.nih.gov/nuccore/NC_024175}{NC\_024175} & \num{62571390} & \num{31285695000} & \num{250} & NO \\
\textit{Oryza glumipatula} & \href{https://trace.ncbi.nlm.nih.gov/Traces/sra/?run=DRR057991}{DRR057991} & \href{https://www.ncbi.nlm.nih.gov/nuccore/NC_027461}{NC\_027461} & \num{23451341} & \num{7035402300} & \num{150} & Yes \\
\textit{Oryza longiglumis} & \href{https://trace.ncbi.nlm.nih.gov/Traces/sra/?run=DRR056658}{DRR056658} & \href{https://www.ncbi.nlm.nih.gov/nuccore/NC_034763}{NC\_034763} & \num{50656466} & \num{15196939800} & \num{150} & Yes \\
\textit{Oryza longistaminata} & \href{https://trace.ncbi.nlm.nih.gov/Traces/sra/?run=DRR058031}{DRR058031} & \href{https://www.ncbi.nlm.nih.gov/nuccore/NC_027462}{NC\_027462} & \num{17927163} & \num{5378148900} & \num{150} & Yes \\
\textit{Oryza meridionalis} & \href{https://trace.ncbi.nlm.nih.gov/Traces/sra/?run=DRR058012}{DRR058012} & \href{https://www.ncbi.nlm.nih.gov/nuccore/NC_016927}{NC\_016927} & \num{19143143} & \num{5742942900} & \num{150} & NO \\
\textit{Oryza meyeriana} & \href{https://trace.ncbi.nlm.nih.gov/Traces/sra/?run=DRR056670}{DRR056670} & \href{https://www.ncbi.nlm.nih.gov/nuccore/NC_034765}{NC\_034765} & \num{41115373} & \num{12334611900} & \num{150} & NO \\
\textit{Oryza minuta} & \href{https://trace.ncbi.nlm.nih.gov/Traces/sra/?run=ERR385916}{ERR385916} & \href{https://www.ncbi.nlm.nih.gov/nuccore/NC_030298}{NC\_030298} & \num{28110596} & \num{6746543040} & \num{120} & Yes \\
\textit{Oryza officinalis} & \href{https://trace.ncbi.nlm.nih.gov/Traces/sra/?run=DRR000604}{DRR000604} & \href{https://www.ncbi.nlm.nih.gov/nuccore/NC_027463}{NC\_027463} & \num{25866032} & \num{5690527040} & \num{110} & Yes \\
\textit{Oryza punctata} & \href{https://trace.ncbi.nlm.nih.gov/Traces/sra/?run=SRR1264539}{SRR1264539} & \href{https://www.ncbi.nlm.nih.gov/nuccore/NC_027676}{NC\_027676} & \num{8000000} & \num{1600000000} & \num{100} & NO \\
\textit{Oryza ridleyi} & \href{https://trace.ncbi.nlm.nih.gov/Traces/sra/?run=DRR056663}{DRR056663} & \href{https://www.ncbi.nlm.nih.gov/nuccore/NC_034764}{NC\_034764} & \num{91020902} & \num{27306270600} & \num{150} & Yes \\
\textit{Oryza rufipogon} & \href{https://trace.ncbi.nlm.nih.gov/Traces/sra/?run=SRR6220521}{SRR6220521} & \href{https://www.ncbi.nlm.nih.gov/nuccore/NC_017835}{NC\_017835} & \num{26511359} & \num{8006430418} & \num{151} & NO \\
\textit{Oryza sativa Indica Group} & \href{https://trace.ncbi.nlm.nih.gov/Traces/sra/?run=SRR553489}{SRR553489} & \href{https://www.ncbi.nlm.nih.gov/nuccore/NC_027678}{NC\_027678} & \num{4956305} & \num{2488065110} & \num{251} & NO \\
\textit{Oryza sativa Japonica Group} & \href{https://trace.ncbi.nlm.nih.gov/Traces/sra/?run=SRR547959}{SRR547959} & \href{https://www.ncbi.nlm.nih.gov/nuccore/NC_001320}{NC\_001320} & \num{5918441} & \num{2971057382} & \num{251} & NO \\
\textit{Oryza sativa} & \href{https://trace.ncbi.nlm.nih.gov/Traces/sra/?run=ERR2696318}{ERR2696318} & \href{https://www.ncbi.nlm.nih.gov/nuccore/NC_031333}{NC\_031333} & \num{6588721} & \num{3947278012} & \num{300} & Yes \\
\textit{Ostreococcus tauri} & \href{https://trace.ncbi.nlm.nih.gov/Traces/sra/?run=SRR4020477}{SRR4020477} & \href{https://www.ncbi.nlm.nih.gov/nuccore/NC_008289}{NC\_008289} & \num{89130001} & \num{17826000200} & \num{100} & NO \\
\textit{Ostrya rehderiana} & \href{https://trace.ncbi.nlm.nih.gov/Traces/sra/?run=SRR8302715}{SRR8302715} & \href{https://www.ncbi.nlm.nih.gov/nuccore/NC_028349}{NC\_028349} & \num{33211296} & \num{8145137640} & \num{123} & NO \\
\textit{Panax ginseng} & \href{https://trace.ncbi.nlm.nih.gov/Traces/sra/?run=SRR5533645}{SRR5533645} & \href{https://www.ncbi.nlm.nih.gov/nuccore/NC_006290}{NC\_006290} & \num{105595137} & \num{37246922641} & \num{176} & Yes \\
\textit{Panax japonicus} & \href{https://trace.ncbi.nlm.nih.gov/Traces/sra/?run=SRR6512791}{SRR6512791} & \href{https://www.ncbi.nlm.nih.gov/nuccore/NC_028703}{NC\_028703} & \num{71312188} & \num{21020452500} & \num{147} & NO \\
\textit{Panax notoginseng} & \href{https://trace.ncbi.nlm.nih.gov/Traces/sra/?run=SRR6512795}{SRR6512795} & \href{https://www.ncbi.nlm.nih.gov/nuccore/NC_026447}{NC\_026447} & \num{7185943} & \num{2100564900} & \num{146} & NO \\
\textit{Panax quinquefolius} & \href{https://trace.ncbi.nlm.nih.gov/Traces/sra/?run=SRR6512793}{SRR6512793} & \href{https://www.ncbi.nlm.nih.gov/nuccore/NC_027456}{NC\_027456} & \num{20952746} & \num{6152880600} & \num{147} & NO \\
\textit{Panax stipuleanatus} & \href{https://trace.ncbi.nlm.nih.gov/Traces/sra/?run=SRR6512796}{SRR6512796} & \href{https://www.ncbi.nlm.nih.gov/nuccore/NC_030598}{NC\_030598} & \num{13610275} & \num{3922500000} & \num{144} & Yes \\
\textit{Panicum capillare} & \href{https://trace.ncbi.nlm.nih.gov/Traces/sra/?run=SRR485871}{SRR485871} & \href{https://www.ncbi.nlm.nih.gov/nuccore/NC_030493}{NC\_030493} & \num{219099842} & \num{43819968400} & \num{100} & Yes \\
\textit{Panicum miliaceum} & \href{https://trace.ncbi.nlm.nih.gov/Traces/sra/?run=SRR6650558}{SRR6650558} & \href{https://www.ncbi.nlm.nih.gov/nuccore/NC_029732}{NC\_029732} & \num{159892965} & \num{80266268430} & \num{251} & Yes \\
\textit{Panicum virgatum} & \href{https://trace.ncbi.nlm.nih.gov/Traces/sra/?run=SRR8440701}{SRR8440701} & \href{https://www.ncbi.nlm.nih.gov/nuccore/NC_015990}{NC\_015990} & \num{92006218} & \num{46187121436} & \num{251} & NO \\
\textit{Papaver somniferum} & \href{https://trace.ncbi.nlm.nih.gov/Traces/sra/?run=SRR6780757}{SRR6780757} & \href{https://www.ncbi.nlm.nih.gov/nuccore/NC_029434}{NC\_029434} & \num{480182430} & \num{252410803808} & \num{263} & NO \\
\textit{Parinari campestris} & \href{https://trace.ncbi.nlm.nih.gov/Traces/sra/?run=SRR1179645}{SRR1179645} & \href{https://www.ncbi.nlm.nih.gov/nuccore/NC_024067}{NC\_024067} & \num{5970365} & \num{1206013730} & \num{101} & NO \\
\textit{Paulownia tomentosa} & \href{https://trace.ncbi.nlm.nih.gov/Traces/sra/?run=SRR6940033}{SRR6940033} & \href{https://www.ncbi.nlm.nih.gov/nuccore/NC_031436}{NC\_031436} & \num{29457173} & \num{8837151900} & \num{150} & NO \\
\textit{Pelargonium citronellum} & \href{https://trace.ncbi.nlm.nih.gov/Traces/sra/?run=SRR576534}{SRR576534} & \href{https://www.ncbi.nlm.nih.gov/nuccore/NC_031194}{NC\_031194} & \num{30750530} & \num{6150106000} & \num{100} & Yes \\
\textit{Pelargonium cotyledonis} & \href{https://trace.ncbi.nlm.nih.gov/Traces/sra/?run=ERR2799534}{ERR2799534} & \href{https://www.ncbi.nlm.nih.gov/nuccore/NC_028052}{NC\_028052} & \num{995842} & \num{300744284} & \num{151} & NO \\
\textit{Pelargonium x hortorum} & \href{https://trace.ncbi.nlm.nih.gov/Traces/sra/?run=SRR576536}{SRR576536} & \href{https://www.ncbi.nlm.nih.gov/nuccore/NC_008454}{NC\_008454} & \num{31358909} & \num{6271781800} & \num{100} & NO \\
\textit{Perilla frutescens} & \href{https://trace.ncbi.nlm.nih.gov/Traces/sra/?run=SRR6940083}{SRR6940083} & \href{https://www.ncbi.nlm.nih.gov/nuccore/NC_030756}{NC\_030756} & \num{34758180} & \num{10427454000} & \num{150} & Yes \\
\textit{Phaseolus vulgaris} & \href{https://trace.ncbi.nlm.nih.gov/Traces/sra/?run=SRR5807695}{SRR5807695} & \href{https://www.ncbi.nlm.nih.gov/nuccore/NC_009259}{NC\_009259} & \num{17399674} & \num{5289500896} & \num{152} & Yes \\
\textit{Phoenix dactylifera} & \href{https://trace.ncbi.nlm.nih.gov/Traces/sra/?run=SRR2518264}{SRR2518264} & \href{https://www.ncbi.nlm.nih.gov/nuccore/NC_013991}{NC\_013991} & \num{89471372} & \num{16641675192} & \num{93} & NO \\
\textit{Phyllostachys edulis} & \href{https://trace.ncbi.nlm.nih.gov/Traces/sra/?run=SRR8245858}{SRR8245858} & \href{https://www.ncbi.nlm.nih.gov/nuccore/NC_015817}{NC\_015817} & \num{196487218} & \num{58946165400} & \num{150} & NO \\
\textit{Physcomitrella patens} & \href{https://trace.ncbi.nlm.nih.gov/Traces/sra/?run=ERR1638191}{ERR1638191} & \href{https://www.ncbi.nlm.nih.gov/nuccore/NC_005087}{NC\_005087} & \num{97932953} & \num{29379885900} & \num{150} & Yes \\
\textit{Picea abies} & \href{https://trace.ncbi.nlm.nih.gov/Traces/sra/?run=ERR1727007}{ERR1727007} & \href{https://www.ncbi.nlm.nih.gov/nuccore/NC_021456}{NC\_021456} & \num{2556898} & \num{644338296} & \num{126} & NO \\
\textit{Picea asperata} & \href{https://trace.ncbi.nlm.nih.gov/Traces/sra/?run=ERR1735493}{ERR1735493} & \href{https://www.ncbi.nlm.nih.gov/nuccore/NC_032367}{NC\_032367} & \num{1862931} & \num{469458612} & \num{126} & Yes \\
\textit{Picea crassifolia} & \href{https://trace.ncbi.nlm.nih.gov/Traces/sra/?run=ERR1735613}{ERR1735613} & \href{https://www.ncbi.nlm.nih.gov/nuccore/NC_032366}{NC\_032366} & \num{1706374} & \num{430006248} & \num{126} & NO \\
\textit{Picea glauca} & \href{https://trace.ncbi.nlm.nih.gov/Traces/sra/?run=SRR869482}{SRR869482} & \href{https://www.ncbi.nlm.nih.gov/nuccore/NC_028594}{NC\_028594} & \num{19568105} & \num{19568105000} & \num{500} & Yes \\
\textit{Picea jezoensis} & \href{https://trace.ncbi.nlm.nih.gov/Traces/sra/?run=ERR1795324}{ERR1795324} & \href{https://www.ncbi.nlm.nih.gov/nuccore/NC_029374}{NC\_029374} & \num{2499244} & \num{629809488} & \num{126} & Yes \\
\textit{Picea sitchensis} & \href{https://trace.ncbi.nlm.nih.gov/Traces/sra/?run=SRR5028182}{SRR5028182} & \href{https://www.ncbi.nlm.nih.gov/nuccore/NC_011152}{NC\_011152} & \num{28752590} & \num{14376295000} & \num{250} & Yes \\
\textit{Pimenta dioica} & \href{https://trace.ncbi.nlm.nih.gov/Traces/sra/?run=SRR5602585}{SRR5602585} & \href{https://www.ncbi.nlm.nih.gov/nuccore/NC_034684}{NC\_034684} & \num{1821150} & \num{1068525583} & \num{293} & Yes \\
\textit{Pinus lambertiana} & \href{https://trace.ncbi.nlm.nih.gov/Traces/sra/?run=SRR2026990}{SRR2026990} & \href{https://www.ncbi.nlm.nih.gov/nuccore/NC_011156}{NC\_011156} & \num{165888035} & \num{49959743872} & \num{151} & Yes \\
\textit{Pinus massoniana} & \href{https://trace.ncbi.nlm.nih.gov/Traces/sra/?run=SRR7666034}{SRR7666034} & \href{https://www.ncbi.nlm.nih.gov/nuccore/NC_021439}{NC\_021439} & \num{6755123} & \num{2026536900} & \num{150} & NO \\
\textit{Pinus sylvestris} & \href{https://trace.ncbi.nlm.nih.gov/Traces/sra/?run=ERR268429}{ERR268429} & \href{https://www.ncbi.nlm.nih.gov/nuccore/NC_035069}{NC\_035069} & \num{162706395} & \num{32541279000} & \num{100} & NO \\
\textit{Pinus taeda} & \href{https://trace.ncbi.nlm.nih.gov/Traces/sra/?run=SRR1049756}{SRR1049756} & \href{https://www.ncbi.nlm.nih.gov/nuccore/NC_021440}{NC\_021440} & \num{4969100} & \num{2534241000} & \num{255} & NO \\
\textit{Piper auritum} & \href{https://trace.ncbi.nlm.nih.gov/Traces/sra/?run=SRR5602592}{SRR5602592} & \href{https://www.ncbi.nlm.nih.gov/nuccore/NC_034697}{NC\_034697} & \num{1951892} & \num{964587352} & \num{247} & NO \\
\textit{Piper nigrum} & \href{https://trace.ncbi.nlm.nih.gov/Traces/sra/?run=SRR5602591}{SRR5602591} & \href{https://www.ncbi.nlm.nih.gov/nuccore/NC_034692}{NC\_034692} & \num{1342936} & \num{797054024} & \num{297} & NO \\
\textit{Pistacia vera} & \href{https://trace.ncbi.nlm.nih.gov/Traces/sra/?run=SRR4453367}{SRR4453367} & \href{https://www.ncbi.nlm.nih.gov/nuccore/NC_034998}{NC\_034998} & \num{62397980} & \num{18719394000} & \num{150} & Yes \\
\textit{Pisum sativum} & \href{https://trace.ncbi.nlm.nih.gov/Traces/sra/?run=SRR6702454}{SRR6702454} & \href{https://www.ncbi.nlm.nih.gov/nuccore/NC_014057}{NC\_014057} & \num{144850725} & \num{29259846450} & \num{101} & NO \\
\textit{Podococcus barteri} & \href{https://trace.ncbi.nlm.nih.gov/Traces/sra/?run=SRR2120220}{SRR2120220} & \href{https://www.ncbi.nlm.nih.gov/nuccore/NC_027276}{NC\_027276} & \num{2453} & \num{671984} & \num{137} & Yes \\
\textit{Populus alba} & \href{https://trace.ncbi.nlm.nih.gov/Traces/sra/?run=SRR3678826}{SRR3678826} & \href{https://www.ncbi.nlm.nih.gov/nuccore/NC_008235}{NC\_008235} & \num{40869133} & \num{8255564866} & \num{101} & Yes \\
\textit{Populus tremula x Populus alba} & \href{https://trace.ncbi.nlm.nih.gov/Traces/sra/?run=SRR1653109}{SRR1653109} & \href{https://www.ncbi.nlm.nih.gov/nuccore/NC_028504}{NC\_028504} & \num{9920342} & \num{2003909084} & \num{101} & NO \\
\textit{Populus tremula} & \href{https://trace.ncbi.nlm.nih.gov/Traces/sra/?run=ERR1735633}{ERR1735633} & \href{https://www.ncbi.nlm.nih.gov/nuccore/NC_027425}{NC\_027425} & \num{3234148} & \num{815005296} & \num{126} & Yes \\
\textit{Populus trichocarpa} & \href{https://trace.ncbi.nlm.nih.gov/Traces/sra/?run=SRR5467392}{SRR5467392} & \href{https://www.ncbi.nlm.nih.gov/nuccore/NC_009143}{NC\_009143} & \num{298329746} & \num{149761532492} & \num{251} & NO \\
\textit{Premna microphylla} & \href{https://trace.ncbi.nlm.nih.gov/Traces/sra/?run=SRR6940036}{SRR6940036} & \href{https://www.ncbi.nlm.nih.gov/nuccore/NC_026291}{NC\_026291} & \num{22624524} & \num{6787357200} & \num{150} & Yes \\
\textit{Primula veris} & \href{https://trace.ncbi.nlm.nih.gov/Traces/sra/?run=SRR1660449}{SRR1660449} & \href{https://www.ncbi.nlm.nih.gov/nuccore/NC_031428}{NC\_031428} & \num{3263725} & \num{1599834089} & \num{245} & NO \\
\textit{Prunus dulcis} & \href{https://trace.ncbi.nlm.nih.gov/Traces/sra/?run=SRR5602582}{SRR5602582} & \href{https://www.ncbi.nlm.nih.gov/nuccore/NC_034696}{NC\_034696} & \num{1285872} & \num{631355142} & \num{245} & NO \\
\textit{Prunus kansuensis} & \href{https://trace.ncbi.nlm.nih.gov/Traces/sra/?run=SRR3138168}{SRR3138168} & \href{https://www.ncbi.nlm.nih.gov/nuccore/NC_023956}{NC\_023956} & \num{72248792} & \num{14449758400} & \num{100} & Yes \\
\textit{Prunus mume} & \href{https://trace.ncbi.nlm.nih.gov/Traces/sra/?run=SRR8240066}{SRR8240066} & \href{https://www.ncbi.nlm.nih.gov/nuccore/NC_023798}{NC\_023798} & \num{45570976} & \num{13671292800} & \num{150} & Yes \\
\textit{Prunus persica} & \href{https://trace.ncbi.nlm.nih.gov/Traces/sra/?run=SRR5073560}{SRR5073560} & \href{https://www.ncbi.nlm.nih.gov/nuccore/NC_014697}{NC\_014697} & \num{9978569} & \num{4989866485} & \num{250} & NO \\
\textit{Psidium guajava} & \href{https://trace.ncbi.nlm.nih.gov/Traces/sra/?run=SRR5812495}{SRR5812495} & \href{https://www.ncbi.nlm.nih.gov/nuccore/NC_033355}{NC\_033355} & \num{20356683} & \num{5129884116} & \num{126} & NO \\
\textit{Punica granatum} & \href{https://trace.ncbi.nlm.nih.gov/Traces/sra/?run=SRR5812494}{SRR5812494} & \href{https://www.ncbi.nlm.nih.gov/nuccore/NC_035240}{NC\_035240} & \num{7814299} & \num{2359918298} & \num{151} & NO \\
\textit{Pyrus pyrifolia} & \href{https://trace.ncbi.nlm.nih.gov/Traces/sra/?run=SRR5196234}{SRR5196234} & \href{https://www.ncbi.nlm.nih.gov/nuccore/NC_015996}{NC\_015996} & \num{33563581} & \num{9782203050} & \num{146} & Yes \\
\textit{Raphanus sativus} & \href{https://trace.ncbi.nlm.nih.gov/Traces/sra/?run=DRR014095}{DRR014095} & \href{https://www.ncbi.nlm.nih.gov/nuccore/NC_024469}{NC\_024469} & \num{513274067} & \num{103681361534} & \num{101} & NO \\
\textit{Rhazya stricta} & \href{https://trace.ncbi.nlm.nih.gov/Traces/sra/?run=ERR351509}{ERR351509} & \href{https://www.ncbi.nlm.nih.gov/nuccore/NC_024292}{NC\_024292} & \num{36920951} & \num{6645771180} & \num{90} & NO \\
\textit{Saccharum officinarum} & \href{https://trace.ncbi.nlm.nih.gov/Traces/sra/?run=SRR7771854}{SRR7771854} & \href{https://www.ncbi.nlm.nih.gov/nuccore/NC_035224}{NC\_035224} & \num{91863015} & \num{46115233530} & \num{251} & Yes \\
\textit{Saccharum spontaneum} & \href{https://trace.ncbi.nlm.nih.gov/Traces/sra/?run=SRR486146}{SRR486146} & \href{https://www.ncbi.nlm.nih.gov/nuccore/NC_034802}{NC\_034802} & \num{63408401} & \num{19149337102} & \num{151} & Yes \\
\textit{Salix purpurea} & \href{https://trace.ncbi.nlm.nih.gov/Traces/sra/?run=SRR3927005}{SRR3927005} & \href{https://www.ncbi.nlm.nih.gov/nuccore/NC_026722}{NC\_026722} & \num{191948456} & \num{57584536800} & \num{150} & Yes \\
\textit{Salix suchowensis} & \href{https://trace.ncbi.nlm.nih.gov/Traces/sra/?run=SRR8255647}{SRR8255647} & \href{https://www.ncbi.nlm.nih.gov/nuccore/NC_026462}{NC\_026462} & \num{53495983} & \num{16048794900} & \num{150} & Yes \\
\textit{Salvia miltiorrhiza} & \href{https://trace.ncbi.nlm.nih.gov/Traces/sra/?run=SRR1735356}{SRR1735356} & \href{https://www.ncbi.nlm.nih.gov/nuccore/NC_020431}{NC\_020431} & \num{215771972} & \num{43585938344} & \num{101} & NO \\
\textit{Sanionia uncinata} & \href{https://trace.ncbi.nlm.nih.gov/Traces/sra/?run=SRR6440975}{SRR6440975} & \href{https://www.ncbi.nlm.nih.gov/nuccore/NC_025668}{NC\_025668} & \num{3961386} & \num{2379651487} & \num{300} & NO \\
\textit{Schizachyrium scoparium} & \href{https://trace.ncbi.nlm.nih.gov/Traces/sra/?run=SRR6442327}{SRR6442327} & \href{https://www.ncbi.nlm.nih.gov/nuccore/NC_035032}{NC\_035032} & \num{1476472} & \num{442941600} & \num{150} & Yes \\
\textit{Sciadopitys verticillata} & \href{https://trace.ncbi.nlm.nih.gov/Traces/sra/?run=ERR2799544}{ERR2799544} & \href{https://www.ncbi.nlm.nih.gov/nuccore/NC_029734}{NC\_029734} & \num{1136673} & \num{343275246} & \num{151} & Yes \\
\textit{Scutellaria baicalensis} & \href{https://trace.ncbi.nlm.nih.gov/Traces/sra/?run=SRR6940088}{SRR6940088} & \href{https://www.ncbi.nlm.nih.gov/nuccore/NC_027262}{NC\_027262} & \num{27930422} & \num{8379126600} & \num{150} & Yes \\
\textit{Scutellaria lateriflora} & \href{https://trace.ncbi.nlm.nih.gov/Traces/sra/?run=SRR5602584}{SRR5602584} & \href{https://www.ncbi.nlm.nih.gov/nuccore/NC_034693}{NC\_034693} & \num{1699048} & \num{843357767} & \num{248} & NO \\
\textit{Sequoia sempervirens} & \href{https://trace.ncbi.nlm.nih.gov/Traces/sra/?run=SRR1951920}{SRR1951920} & \href{https://www.ncbi.nlm.nih.gov/nuccore/NC_030372}{NC\_030372} & \num{2354579} & \num{1412747400} & \num{300} & NO \\
\textit{Sesamum indicum} & \href{https://trace.ncbi.nlm.nih.gov/Traces/sra/?run=SRR3407090}{SRR3407090} & \href{https://www.ncbi.nlm.nih.gov/nuccore/NC_016433}{NC\_016433} & \num{20637541} & \num{5159385250} & \num{125} & NO \\
\textit{Setaria italica} & \href{https://trace.ncbi.nlm.nih.gov/Traces/sra/?run=ERR2696316}{ERR2696316} & \href{https://www.ncbi.nlm.nih.gov/nuccore/NC_022850}{NC\_022850} & \num{65844392} & \num{19753317600} & \num{150} & Yes \\
\textit{Setaria viridis} & \href{https://trace.ncbi.nlm.nih.gov/Traces/sra/?run=SRR4063259}{SRR4063259} & \href{https://www.ncbi.nlm.nih.gov/nuccore/NC_028075}{NC\_028075} & \num{106716486} & \num{33508976604} & \num{157} & Yes \\
\textit{Silene latifolia} & \href{https://trace.ncbi.nlm.nih.gov/Traces/sra/?run=SRR7040334}{SRR7040334} & \href{https://www.ncbi.nlm.nih.gov/nuccore/NC_016730}{NC\_016730} & \num{284434421} & \num{71108605250} & \num{125} & NO \\
\textit{Solanum berthaultii} & \href{https://trace.ncbi.nlm.nih.gov/Traces/sra/?run=SRR5349620}{SRR5349620} & \href{https://www.ncbi.nlm.nih.gov/nuccore/NC_034951}{NC\_034951} & \num{30261020} & \num{7565255000} & \num{125} & NO \\
\textit{Solanum commersonii} & \href{https://trace.ncbi.nlm.nih.gov/Traces/sra/?run=SRR5349609}{SRR5349609} & \href{https://www.ncbi.nlm.nih.gov/nuccore/NC_028069}{NC\_028069} & \num{31268502} & \num{7817125500} & \num{125} & NO \\
\textit{Solanum habrochaites} & \href{https://trace.ncbi.nlm.nih.gov/Traces/sra/?run=DRR098098}{DRR098098} & \href{https://www.ncbi.nlm.nih.gov/nuccore/NC_026879}{NC\_026879} & \num{95529750} & \num{17768533500} & \num{93} & NO \\
\textit{Solanum lycopersicum} & \href{https://trace.ncbi.nlm.nih.gov/Traces/sra/?run=DRR022700}{DRR022700} & \href{https://www.ncbi.nlm.nih.gov/nuccore/AC_000188}{AC\_000188} & \num{4192871} & \num{2104821242} & \num{251} & Yes \\
\textit{Solanum melongena} & \href{https://trace.ncbi.nlm.nih.gov/Traces/sra/?run=DRR014074}{DRR014074} & \href{https://www.ncbi.nlm.nih.gov/nuccore/NC_030207}{NC\_030207} & \num{173060640} & \num{34958249280} & \num{101} & NO \\
\textit{Solanum peruvianum} & \href{https://trace.ncbi.nlm.nih.gov/Traces/sra/?run=SRR1291239}{SRR1291239} & \href{https://www.ncbi.nlm.nih.gov/nuccore/NC_026881}{NC\_026881} & \num{12772339} & \num{2449431901} & \num{96} & Yes \\
\textit{Solanum pimpinellifolium} & \href{https://trace.ncbi.nlm.nih.gov/Traces/sra/?run=SRR074949}{SRR074949} & \href{https://www.ncbi.nlm.nih.gov/nuccore/NC_026882}{NC\_026882} & \num{195678316} & \num{39527019832} & \num{101} & Yes \\
\textit{Solanum tuberosum} & \href{https://trace.ncbi.nlm.nih.gov/Traces/sra/?run=ERR3148763}{ERR3148763} & \href{https://www.ncbi.nlm.nih.gov/nuccore/NC_008096}{NC\_008096} & \num{806926} & \num{243691652} & \num{151} & NO \\
\textit{Sorghum bicolor} & \href{https://trace.ncbi.nlm.nih.gov/Traces/sra/?run=SRR3927127}{SRR3927127} & \href{https://www.ncbi.nlm.nih.gov/nuccore/NC_008602}{NC\_008602} & \num{30065772} & \num{9079863144} & \num{151} & NO \\
\textit{Sorghum timorense} & \href{https://trace.ncbi.nlm.nih.gov/Traces/sra/?run=SRR424217}{SRR424217} & \href{https://www.ncbi.nlm.nih.gov/nuccore/NC_023800}{NC\_023800} & \num{200900821} & \num{40180164200} & \num{100} & Yes \\
\textit{Spirodela polyrhiza} & \href{https://trace.ncbi.nlm.nih.gov/Traces/sra/?run=SRR7548934}{SRR7548934} & \href{https://www.ncbi.nlm.nih.gov/nuccore/NC_015891}{NC\_015891} & \num{87602747} & \num{26401985035} & \num{151} & Yes \\
\textit{Sporobolus michauxianus} & \href{https://trace.ncbi.nlm.nih.gov/Traces/sra/?run=SRR4434178}{SRR4434178} & \href{https://www.ncbi.nlm.nih.gov/nuccore/NC_029416}{NC\_029416} & \num{183439649} & \num{55031894700} & \num{150} & NO \\
\textit{Stachys byzantina} & \href{https://trace.ncbi.nlm.nih.gov/Traces/sra/?run=SRR3170744}{SRR3170744} & \href{https://www.ncbi.nlm.nih.gov/nuccore/NC_029825}{NC\_029825} & \num{212918} & \num{64301236} & \num{151} & Yes \\
\textit{Stachys chamissonis} & \href{https://trace.ncbi.nlm.nih.gov/Traces/sra/?run=SRR3170745}{SRR3170745} & \href{https://www.ncbi.nlm.nih.gov/nuccore/NC_029822}{NC\_029822} & \num{2570411} & \num{776264122} & \num{151} & Yes \\
\textit{Stachys coccinea} & \href{https://trace.ncbi.nlm.nih.gov/Traces/sra/?run=SRR3170746}{SRR3170746} & \href{https://www.ncbi.nlm.nih.gov/nuccore/NC_029823}{NC\_029823} & \num{292594} & \num{88363388} & \num{151} & Yes \\
\textit{Stachys sylvatica} & \href{https://trace.ncbi.nlm.nih.gov/Traces/sra/?run=SRR3170747}{SRR3170747} & \href{https://www.ncbi.nlm.nih.gov/nuccore/NC_029824}{NC\_029824} & \num{592661} & \num{178983622} & \num{151} & NO \\
\textit{Taraxacum kok-saghyz} & \href{https://trace.ncbi.nlm.nih.gov/Traces/sra/?run=SRR8185394}{SRR8185394} & \href{https://www.ncbi.nlm.nih.gov/nuccore/NC_032057}{NC\_032057} & \num{21183799} & \num{11049748919} & \num{261} & NO \\
\textit{Taxodium distichum} & \href{https://trace.ncbi.nlm.nih.gov/Traces/sra/?run=ERR2799546}{ERR2799546} & \href{https://www.ncbi.nlm.nih.gov/nuccore/NC_034941}{NC\_034941} & \num{879574} & \num{265631348} & \num{151} & Yes \\
\textit{Taxus baccata} & \href{https://trace.ncbi.nlm.nih.gov/Traces/sra/?run=ERR268425}{ERR268425} & \href{https://www.ncbi.nlm.nih.gov/nuccore/NC_035066}{NC\_035066} & \num{172310721} & \num{34462144200} & \num{100} & Yes \\
\textit{Tectona grandis} & \href{https://trace.ncbi.nlm.nih.gov/Traces/sra/?run=SRR6940065}{SRR6940065} & \href{https://www.ncbi.nlm.nih.gov/nuccore/NC_020098}{NC\_020098} & \num{24194338} & \num{7258301400} & \num{150} & Yes \\
\textit{Tetradesmus obliquus} & \href{https://trace.ncbi.nlm.nih.gov/Traces/sra/?run=ERR1683652}{ERR1683652} & \href{https://www.ncbi.nlm.nih.gov/nuccore/NC_008101}{NC\_008101} & \num{100944122} & \num{20390712644} & \num{101} & NO \\
\textit{Themeda triandra} & \href{https://trace.ncbi.nlm.nih.gov/Traces/sra/?run=SRR7529014}{SRR7529014} & \href{https://www.ncbi.nlm.nih.gov/nuccore/NC_035016}{NC\_035016} & \num{91131777} & \num{45565888500} & \num{250} & Yes \\
\textit{Theobroma cacao} & \href{https://trace.ncbi.nlm.nih.gov/Traces/sra/?run=SRR5602583}{SRR5602583} & \href{https://www.ncbi.nlm.nih.gov/nuccore/NC_014676}{NC\_014676} & \num{470765} & \num{277504475} & \num{295} & NO \\
\textit{Thlaspi arvense} & \href{https://trace.ncbi.nlm.nih.gov/Traces/sra/?run=SRR1034659}{SRR1034659} & \href{https://www.ncbi.nlm.nih.gov/nuccore/NC_034362}{NC\_034362} & \num{10761690} & \num{5402368380} & \num{251} & NO \\
\textit{Trifolium subterraneum} & \href{https://trace.ncbi.nlm.nih.gov/Traces/sra/?run=DRR032042}{DRR032042} & \href{https://www.ncbi.nlm.nih.gov/nuccore/NC_011828}{NC\_011828} & \num{25964880} & \num{15630857760} & \num{301} & Yes \\
\textit{Triticum aestivum} & \href{https://trace.ncbi.nlm.nih.gov/Traces/sra/?run=SRR5893651}{SRR5893651} & \href{https://www.ncbi.nlm.nih.gov/nuccore/NC_002762}{NC\_002762} & \num{197767716} & \num{104816889480} & \num{265} & NO \\
\textit{Triticum monococcum} & \href{https://trace.ncbi.nlm.nih.gov/Traces/sra/?run=SRR384895}{SRR384895} & \href{https://www.ncbi.nlm.nih.gov/nuccore/NC_021760}{NC\_021760} & \num{157945836} & \num{31905058872} & \num{101} & NO \\
\textit{Triticum urartu} & \href{https://trace.ncbi.nlm.nih.gov/Traces/sra/?run=SRR4010671}{SRR4010671} & \href{https://www.ncbi.nlm.nih.gov/nuccore/NC_021762}{NC\_021762} & \num{123772698} & \num{63619166772} & \num{257} & NO \\
\textit{Urochloa ruziziensis} & \href{https://trace.ncbi.nlm.nih.gov/Traces/sra/?run=SRR6816710}{SRR6816710} & \href{https://www.ncbi.nlm.nih.gov/nuccore/NC_030068}{NC\_030068} & \num{154533231} & \num{46669035762} & \num{151} & NO \\
\textit{Utricularia reniformis} & \href{https://trace.ncbi.nlm.nih.gov/Traces/sra/?run=SRR5349713}{SRR5349713} & \href{https://www.ncbi.nlm.nih.gov/nuccore/NC_029719}{NC\_029719} & \num{2186073} & \num{1157291061} & \num{265} & NO \\
\textit{Vaccinium macrocarpon} & \href{https://trace.ncbi.nlm.nih.gov/Traces/sra/?run=SRR1276173}{SRR1276173} & \href{https://www.ncbi.nlm.nih.gov/nuccore/NC_019616}{NC\_019616} & \num{30540329} & \num{9223179358} & \num{151} & NO \\
\textit{Vicia sativa} & \href{https://trace.ncbi.nlm.nih.gov/Traces/sra/?run=ERR413103}{ERR413103} & \href{https://www.ncbi.nlm.nih.gov/nuccore/NC_027155}{NC\_027155} & \num{10232390} & \num{2046478000} & \num{100} & NO \\
\textit{Vigna unguiculata} & \href{https://trace.ncbi.nlm.nih.gov/Traces/sra/?run=SRR7125688}{SRR7125688} & \href{https://www.ncbi.nlm.nih.gov/nuccore/NC_018051}{NC\_018051} & \num{30233396} & \num{9070018800} & \num{150} & Yes \\
\textit{Vitis aestivalis} & \href{https://trace.ncbi.nlm.nih.gov/Traces/sra/?run=SRR5891909}{SRR5891909} & \href{https://www.ncbi.nlm.nih.gov/nuccore/NC_029454}{NC\_029454} & \num{17094973} & \num{5128491900} & \num{150} & Yes \\
\textit{Vitis amurensis} & \href{https://trace.ncbi.nlm.nih.gov/Traces/sra/?run=SRR5891950}{SRR5891950} & \href{https://www.ncbi.nlm.nih.gov/nuccore/NC_031383}{NC\_031383} & \num{21559227} & \num{6467768100} & \num{150} & NO \\
\textit{Vitis rotundifolia} & \href{https://trace.ncbi.nlm.nih.gov/Traces/sra/?run=SRR5627788}{SRR5627788} & \href{https://www.ncbi.nlm.nih.gov/nuccore/NC_023790}{NC\_023790} & \num{49224878} & \num{14767463400} & \num{150} & NO \\
\textit{Vitis vinifera} & \href{https://trace.ncbi.nlm.nih.gov/Traces/sra/?run=SRR7160359}{SRR7160359} & \href{https://www.ncbi.nlm.nih.gov/nuccore/NC_007957}{NC\_007957} & \num{8416372} & \num{8416372000} & \num{500} & NO \\
\textit{Viviania marifolia} & \href{https://trace.ncbi.nlm.nih.gov/Traces/sra/?run=ERR2799536}{ERR2799536} & \href{https://www.ncbi.nlm.nih.gov/nuccore/NC_023259}{NC\_023259} & \num{1050522} & \num{317257644} & \num{151} & NO \\
\textit{Wisteria floribunda} & \href{https://trace.ncbi.nlm.nih.gov/Traces/sra/?run=SRR1265941}{SRR1265941} & \href{https://www.ncbi.nlm.nih.gov/nuccore/NC_027677}{NC\_027677} & \num{9382854} & \num{1895336508} & \num{101} & Yes \\
\textit{Wollemia nobilis} & \href{https://trace.ncbi.nlm.nih.gov/Traces/sra/?run=SRR1927951}{SRR1927951} & \href{https://www.ncbi.nlm.nih.gov/nuccore/NC_027235}{NC\_027235} & \num{116590} & \num{59624823} & \num{256} & NO \\
\textit{Zea mays} & \href{https://trace.ncbi.nlm.nih.gov/Traces/sra/?run=SRR5826129}{SRR5826129} & \href{https://www.ncbi.nlm.nih.gov/nuccore/NC_001666}{NC\_001666} & \num{172800442} & \num{89856229840} & \num{260} & Yes \\
\end{longtable}

\newpage
{\tiny
\captionof{table}{Docker images used in our benchmark setup.}
\label{tab:dockerimages_suppl}
\centering
\begin{tabular}{ccc}
        \hline
          Tool & Image name and tag & SHA256 Checksum   \\ \hline
          \ce & \dockerce & \dockercesha \\
          \cassp & \dockercassp & \dockercasspsha \\
          \fp & \dockerfp & \dockerfpsha \\
          \go & \dockergo & \dockergosha \\
          \ioga & \dockerioga & \dockeriogasha \\
          \np & \dockernp & \dockernpsha \\
          \oa & \dockeroa & \dockeroasha \\ \hline
      \end{tabular}
}
\end{landscape}
\end{document}